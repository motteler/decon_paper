
Spectra of the earth's thermal emission as measured by the {\airs},
{\cris}, and {\iasi} hyper-spectral sounders is a significant part
of the long term climate record.  These instruments have broadly
similar spatial sampling, spectral resolution, channel response
functions, and band spans.  However the channel response functions
vary in detail, leading to significant differences in observed
spectra.  For applications such as calibaration and validation,
retrievals, and the construction of a long term climate record we
would like to work with single set of spectral response functions.
This is can be done by translating channel radiances from one
sounder to another, including simulation of the response functions
of the translation target.  

We make regular use such of translations from {\airs} to 
{\cris} and {\iasi} to {\cris}, and have implemented and tested
{\iasi} to {\airs} and {\cris} to {\airs} translations as well.  
Our translation from {\airs} to {\cris} has some novel features.
{\airs} is a grating spectrometer with a distinct response function
for each channel, while {\cris} is a Michaelson interferometer with
a sinc response function after calibration and corrections.  We use
our detailed knowledge of the {\airs} spectral response functions to
deconvolve {\airs} channel radiances to a resolution enhanced
intermediate representation.  This is reconvolved to {\cris} or
other instrument specifications.  The resulting translation is shown
to be more accurate than interpolation or conventional regression.

