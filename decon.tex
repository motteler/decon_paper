\documentclass[11pt]{article}
\usepackage{graphicx}
\usepackage{placeins}

\input{crisdefs.tex}

\title{AIRS Deconvolution and Translation \\
  from the AIRS to CrIS IR Sounders \\
  \vspace{3mm}
  {****} DRAFT {****}\\
}

\author{Howard E.~Motteler \\
  L.~Larrabee Strow \\
  \\
  UMBC Atmospheric Spectroscopy Lab \\
  Joint Center for Earth Systems Technology \\
}

\date{\today}
\begin{document}
\maketitle

\section{Introduction}

Upwelling infrared radiation as measured by the {\airs} \cite{airs1}
and {\cris} \cite{cris1,cris2} sounders is a significant part of the
long term climate record.  We would like to treat this as a single
data set and often want to compare radiances, for example in the
analysis of simultaneous nadir overpasses (SNOs) for sounder
calibration or validation.  However the instruments have different
spectral resolutions, channel response functions, and band spans.
As a step in addressing this problem we consider the translation of
channel radiances from {\airs} to standard resolution {\cris}.

In addition to {\airs} to {\cris} we also make regular use of an
{\iasi} to {\cris} translation and have implemented and tested
{\iasi} to {\airs} and {\cris} to {\airs} translations as well.  
The translations from {\iasi} includes deapodization (a form of
deconvolution) before reconvolution to the translation target, and
work very well.  In order of most to least accurate in comparison
with calculated reference truth we have {\iasi} to {\cris}, {\iasi}
to {\airs}, {\airs} to {\cris}, and finally {\cris} to {\airs}
\cite{deconATBD}.  But aside from {\airs} to {\cris}, the methods
used are conventional.

Translation from {\airs} to {\cris} involves more that basic
resampling.  {\airs} is a grating spectrometer with a distinct
response function for each channel determined by the focal plane
geometery, while {\cris} is a Michaelson interferometer with a sinc
response function, after calibration and corrections.  In section
\ref{decon} we show how to take advantage of our detailed knowledge
of the {\airs} spectral response functions (SRFs) and their overlap
to deconvolve channel radiances to a resolution-enhanced
intermediate representation, typically a $0.1$~\wn\ grid, the
approximate resolution of the tabulated {\airs} SRFs.  

This intermediate representation can then be reconvolved to an
alternate instrument specification.  Section \ref{airs2cris} gives
details and validation tests for the {\airs} to {\cris} translation,
and section \ref{airsL1d} for translation to an idealized grating
model.  For both cases deconvolution followed by reconvolution is
shown to work significantly better than conventional interpolation.
Both methods can be further improved with a statistical correcion.
In section \ref{dregr} we consider a purely statistical approach to
such translations.  Section \ref{appcon} discusses applications and
related and future work.

\FloatBarrier
\section{AIRS Deconvolution}
\label{decon}

The {\airs} spectral response functions model channel response as a
function of frequency and associate channels with nominal center
frequencies.  Each {\airs} channel $i$ has an associated spectral
response function or {\srf} $\sigma_i(v)$ such that the channel
radiance $c_i = \int \sigma_i(v)r(v)\,dv$, where $r$ is radiance at
frequency $v$.  The center or peak of $\sigma_i$ is the nominal
channel frequency.

\begin{figure} % source plot_SRF2.m
  \centering
  \includegraphics[height=7.5cm]{figures/airs_sample_srfs.pdf}
  \caption{sample {\airs} spectral response functions from the low
    and high ends of the band.   The dashed line is a generalized
    Gaussian function.}
  \label{srfs1}
\end{figure}

\begin{figure} % source plot_SRF2.m
  \centering
  \includegraphics[height=7.5cm]{figures/airs_L1c_res.pdf}
  \caption{{\airs} L1c channel spacing and derived resolving
    power.}
  \label{chan1}
\end{figure}

Figure \ref{srfs1} shows typical {\airs} SRFs from the low and high
ends of the band.  Note the significant overlap in the wings.  This
can allow for a deconvolution to recover resolution beyond that of
the response functions considered individually.  The SRFs are not
necessarily symmetrical, especially at the high end of the band.
The dashed line on top of the third SRF is a fit of a generalized
Gaussian, which we consider in more detail later in this section.
Figure \ref{chan1} shows channel spacing and resolving power for the
{\airs} L1c channel set \cite{airs1c}.  The variable channel spacing
and resolving power are due to the modular structure of the focal
plane.  Although not entirely regular---that is, not a simple
function of frequency---the L1c channel set is more regular than the
L1b channel set from which it is derived, and we mainly consider the
L1c set here.

Suppose we have $n$ channels and a frequency grid $\vec v$ of 
$k$ points spanning the union of the domains of the functions
$\sigma_i$.  The grid step size for our applications is often 0.0025
{\wn}, the default resolution for upwelling radiance calculations
with \cite{kcarta1}.  Let $S_k$ be an $n\times k$ array such that
$s_{i,j} = \sigma_i(v_j)/w_i$, where $w_i = \sum_j \sigma_i(v_j)$,
that is where row $i$ is $\sigma_i(v)$ tabulated at the grid $\vec
v$ and normalized so the row sum is 1.  If the channel centers are
in increasing order $S_k$ is banded, and if they are not too close
(as is the case for a few of the L1b channels) the rows are linearly
independent.  $S_k$ is a linear transform whose domain is radiance
at the grid $\vec v$ and whose range is channel radiances.  If $r$
is radiance at the grid $\vec v$, then $c = S_k r$ gives a good
approximation of the channel radiances $c_i =
\int\sigma_i(v)r(v)\,dv$.  In practice this is how we convolve
kcarta or other high resolution calculated radiances to get {\airs}
channel radiances, for example for reference truth or ``true
{\airs}'' for the tests shown here.

% We construct $S_k$ either explicitly or implicitly from the
% {\airs} {\srf} tabulations.  The matrix $S_k$ in the former case
% is large but manageable with a banded or sparse representation.

For the {\airs} to {\cris} and other translations we are mainly
interested in the transform $S_b$ for {\srf}s at an intermediate
resolution, typically $0.1~\wn$.  This is the approximate resolution
of the {\srf} measurements and convenient for reconvolution to the
{\cris} user grid.  So let $\vec v_b = v_1,v_2,\ldots,v_m$ be a $0.1
\wn$ grid spanning the domains of the functions $\sigma_i$.  Similar
to $S_k$, let $S_b$ be an $n\times m$ array where row $i$ is
$\sigma_i(v)$ tabulated at the $\vec v_b$ grid, with rows normalized
to~1.  If $r$ is radiance at the $\vec v_b$ grid, then $c = S_b r$
is still a reasonable approximation of $\int\sigma_i(v)r(v)\,dv$.

For our application we want to start with $c$ and find $r$, that is
to deconvolve $c$ by solving $S_b r = c$ for $r$.  Since $m < k$,
the system is underdetermined, but we can take the Moore-Penrose
pseudoinverse \cite{pinv} of $S_b$ to get $r_0 = S_b^{-1} c$.  This
gives a minimal solution, in the sense that $||r_0||_2 \le
||r_j||_2$ for all $r_j$ satisfying $S_b r_j = c$.  The condition
number for $S_b$ built from the L1c channels is
$||S_b||_2||S_b^{-1}||_2 = 115$, which is tolerable.

Although our main goal is to reconvolve the $0.1~\wn$ intermediate
representation to the {\cris} or other user grids, we first compare
the deconvolved radiances with reference truth from a direct
convolution to intermediate grid.  The choice of response functions
for this direct convolution is not obvious, since the deconvolution
is undoing---at least to some extent---the effects of the {\airs}
SRF convolutions.  We chose a generalized Gaussian of the form
\[w = \exp\left(-\left(\frac{(x - v_0)^2}{2c^2}\right)^{1.5}\right) \]
where $c=\fwhm / 2.355$ and $v_0$ is the desired channel center.
The exponent $1.5$ was chosen to give an approximate match to
{\airs} SRFs with the same {\fwhm} and channel centers, though
without the fine structure and variation of the latter.  Figure
\ref{srfs1} shows two such functions paired with {\airs} SRFs with
the same {\fwhm} and centers.  We used this function for the
0.1~\wn\ intermediate grid with ${\fwhm} = v_i / 2000$ where $v_i$
are the grid frequencies.  This represents a hypothetical grating
spectrometer with a resolving power of 2000, oversampled to the
0.1~\wn\ grid.  The residual was roughly minimized with resolving
power of 2000, as shown here.  We also tried the generalized
Gaussian with a fixed \fwhm\ for values $0.4$, $0.6$, and $0.8$ and
a sinc basis with a spacing of $0.2$~\wn, all of which gave larger
residuals.

\begin{figure} % source decon_test1.m 
  \centering
  \includegraphics[height=7.5cm]{figures/airs_decon_zoom.pdf}
  \caption{details from fitting profile 1 for kcarta, direct
    convolution to the $0.1$~\wn\ grid (``gauss''), deconvolved
    {\airs}, and true {\airs}.}
  \label{dzoom}
\end{figure}

\begin{figure} % source decon_test1.m
  \centering
  \includegraphics[height=7.5cm]{figures/airs_decon_spec.pdf}
  \caption{spectra from fitting profile 1 for direct convolution to
    the $0.1$~\wn\ grid (``gauss'') and deconvolved {\airs}}
  \label{dspec}
\end{figure}

% \begin{figure} % source decon_test1.m
%   \centering
%   \includegraphics[height=7.5cm]{figures/airs_decon_diff.pdf}
%   \caption{mean and standard deviation over the 49 fitting profiles
%     for the L1c deconvolution minus direct convolution to the
%     $0.1$~\wn\ intermediate grid.  The residuals are too large to use
%     the deconvolved radiances directly.}
%   \label{ddiff}
% \end{figure}

The {\airs} deconvolution gives a modest resolution enhancement, at
the cost of added artifacts and noise.  Figure \ref{dzoom} shows
details of kcarta, direct convolution to the $0.1$~\wn\ grid
(``gauss''), deconvolution, and AIRS spectra for fitting profile~1
\cite{sarta1,sarta2}.  In the first subplot we see the deconvolution
is capturing some of the fine structure in the kcarta data that is
present in the direct convolution but not in the AIRS data.  In the
second subplot we see the deconvolution (and direct convolution)
resolving a pair of close lines that are not resolved at the {\airs}
L1c resolution.  But we also see some ringing that is not present in
the direct convolution.  Figure \ref{dspec} shows the full spectra
from fitting profile~1, along with sample details from the low and
high ends of the band, for the deconvolution and direct convolution
to the intermediate grid.  In the details we see some overshoot and
ringing in the deconvolution.  But as noted we do not propose using
the deconvolved radiances directly, they are an intermediate step in
reconvolution to a lower resolution.

% Figure \ref{ddiff} shows the mean and standard deviation of the
% difference of the deconvolved minus the directly convolved radiances
% for all 49 fitting profiles.  The residuals are large but mainly
% significant for understanding limitations of the deconvolution.

% The residuals can be reduced dramatically by reconvolving the
% $0.1$~\wn\ intermediate grid to a lower resolution.  We consider
% this for convolution to the {\cris} user grid in the next section.

\begin{figure} % source plot_Binv.m
  \centering
  \includegraphics[height=7.5cm]{figures/airs_decon_basis.pdf}
  \caption{sample adjacent rows for the deconvolution and L1c to L1d
    transforms}
  \label{dbasis}
\end{figure}

Figure \ref{dbasis} shows a pair of typical adjacent rows of the
deconvolution matrix $S_b^{-1}$\, in the first subplot.  Row $i$ of
$S_b^{-1}$ is the weights applied to L1c channel radiances to
synthesize the deconvolved radiance $r_i$ at the intermediate grid
frequency $v_i$.  The oscillation shows we are taking the closest
AIRS channel, subtracting weighted values for channels $\pm 1$ step
away, adding weighted values for channels $\pm 2$ steps away, and so
on, with the weights decreasing quickly as we move away from $v_i$,
with eight to ten L1c channels making a significant contribution to
each deconvolution grid point.

The second subplot shows four adjacent rows of the matrix 
$S_d \cdot S_b^{-1}$, which takes L1c to L1d channel radiances.
(The L1d radiances are discussed in a later section; here they are
of interest mainly as a typical reconvolution.)  Both matrices are
banded but the bands are narrower in the second, with three to five
L1c channels contributing significantly to each L1d channel.  
The range of influence is significant since for example we may want
to see which L1d channels are derived in part from the subset of
synthetic L1c channels.

\FloatBarrier
\section{AIRS to CrIS translation}
\label{airs2cris}

[need intro sentence]


Given {\airs} deconvolution to the $0.1~\wn$ intermediate grid,
reconvolution to the {\cris} user grid is straightforward.  For the
{\cris} standard resolution mode the channel spacing is $0.625~\wn$
for the LW, $1.25~\wn$ for the MW, and $2.5~\wn$ for the SW bands.
For each {\cris} band, we (1) find the {\airs} and {\cris} band
intersection, (2) apply a bandpass filter to the deconvolved {\airs}
radiances restricting them to the intersection, with a rolloff
outside the intersection, and (3) reconvolve the filtered spectra to
the {\cris} user grid with a zero-filled double Fourier transform.

Translations are tested by comparison with calculated reference
truth.  We start with a set of atmospheric profiles and calculate
upwelling radiance at a $0.0025~\wn$ grid with kcarta \cite{kcarta1}
over a band spanning the domains of the {\airs} and {\cris} response
functions.  ``True {\airs}'' is calculated by convolving the kcarta
radiances with {\airs} SRFs and ``true {\cris}'' by convolving
kcarta radiances to a sinc basis at the {\cris} user-grid.  True
{\airs} is then translated to {\cris} to get ``{\airs} {\cris}'',
and this is compared with true {\cris}.  Figure~\ref{specLW} shows
sample spectra for true {\airs}, deconvolved {\airs}, true {\cris}
and {\airs} {\cris}.  The difference between true {\cris} and
{\airs} {\cris} is hard to see at this level of detail, and for the
remainder of this paper we will mainly be looking at differences.

For most tests we use a set of 49 fitting profiles spanning a wide
range of clear atmospheric conditions, initially chosen for testing
radiative transfer codes \cite{sarta1,sarta2}.  The set is largely
uncorrelated, in the sense that reducing the reconstruction residual
to 0.02~K for requires 48 left-singular vectors.  (Details of this
correlation measure are givein in the appendix.)  For statistical
correction and direct regression (discussed later) we also use a set
of 7377 radiances calculated from mostly cloudy AIRS profiles
spanning several consecutive days, as the dependent set.  This set
is moderately correlated; reducing the reconstruction residual to
0.02~K requires 260 left-singular vectors.  For all cases considered
here the 49 profile set gives a stricter test than splitting the
7377 profile set into dependent and independent sets as residuals
with the 49-profile set are consistently larger.

Figures \ref{diffLW}, \ref{diffMW}, and \ref{diffSW} show the mean
and standard deviation of true {\cris} minus {\airs} {\cris} for the
49 fitting profiles, with and without Hamming apodization, for each
of the {\cris} bands.  Figure \ref{meanAll} summarizes these results
for Hamming apodized radiances.  The residual has a high frequency
component with a period of 2 channel steps that is significantly
reduced by the apodization.  The constant or DC bias (the mean of
the residuals over frequency) is very close to zero for the apodized
residuals.

% $0.002$~K for the LW, $-0.005$~K for the MW, and $0.001$~K for the
% SW.

\begin{figure} % source a2cris_test1
  \centering
  \includegraphics[height=7.5cm]{figures/a2cris_spec_LW.pdf}
  \caption{true {\airs}, deconvolved {\airs}, true {\cris}, and
    {\airs} {\cris}}
  \label{specLW}
\end{figure}

\begin{figure} % source a2cris_test1
  \centering
  \includegraphics[height=7.5cm]{figures/a2cris_diff_LW.pdf}
  \caption{Mean and standard deviation of unapodized and Hamming
    apodized {\airs} {\cris} minus true {\cris}, for the {\cris} LW
    band}
  \label{diffLW}
\end{figure}

\begin{figure} % source a2cris_test1
  \centering
  \includegraphics[height=7.5cm]{figures/a2cris_diff_MW.pdf}
  \caption{Mean and standard deviation of unapodized and Hamming
    apodized {\airs} {\cris} minus true {\cris}, for the {\cris} MW
    band}
  \label{diffMW}
\end{figure}

\begin{figure} % source a2cris_test1
  \centering
  \includegraphics[height=7.5cm]{figures/a2cris_diff_SW.pdf}
  \caption{Mean and standard deviation of unapodized and Hamming
    apodized {\airs} {\cris} minus true {\cris}, for the {\cris} SW
    band}
  \label{diffSW}
\end{figure}

\begin{figure} % source a2cris_test1
  \centering
  \includegraphics[height=7.5cm]{figures/a2cris_diff_all.pdf}
  \caption{Mean of apodized residuals for all three {\cris} bands}
  \label{meanAll}
\end{figure}

There is some regularity in the residual, including an oscillation
with period two channel steps.  Up to this point there is no
statistical component to our translation, beyond finding a good test
set for validation.  We feel it is important to be clear about any
steps that require statistical fitting.  That said, a simple linear
correction can give a significantly further reduction in the
validation residual.  For the statistical tests we use the set of
7377 mostly cloudy AIRS profiles as the dependent set and the 49
profile set the independent or test set.

We compare three such corrections.  These are done with a separate
regression for each {\cris} channel, and so introduce no
cross-correlations.  Let $\Ttc_i$ be true {\cris} and $\Tac_i$
{\airs} to {\cris} brightness temperatures for {\cris} channel $i$,
both from the dependent set.  For the bias test we subtract the mean
residual from the dependent set.  For the linear test we find $a_i$
and $b_i$ to minimize $||a_i\,\Tac_i + b_i - \Ttc_i||_2$ and for the
quadratic test weights $c_i$, $a_i$ and $b_i$ to minimize
$||c_i\,(\Tac_i)^2 + a_i\,\Tac_i + b_i - \Ttc_i||_2$.

Figure \ref{statLW} is a comparison of bias, linear, and quadratic
corrections for the LW band.  The linear and quadratic corrections
are nearly identical, and the quadratic coefficient is very close 
to zero.  Figure \ref{coefLW} shows the weights for the linear fits
from figure \ref{statLW}.  The $a$ weight is very close to 1 and the
$b$ weight to the bias.  Figures \ref{statMW} and \ref{statSW} show
the linear correction giving a similar improvement in the MW and a
small improvement in the SW, where the quadratic correction is
noticably worse.  Figure \ref{statall} summarized results for the
linear correction, pairing these with the apodized uncorrected
residuals.

% Note that convolution, deconvolution, and apodization are done
% with radiances while spectra are presented and statistics done
% after translation to brightness temperatures.

\begin{figure} % source a2cris_regr2.m
  \centering
  \includegraphics[height=7.5cm]{figures/a2cris_regr_LW.pdf}
  \caption{Mean and standard deviation of LW corrected apodized
    residuals}
  \label{statLW}
\end{figure}

\begin{figure} % source a2cris_regr2.m
  \centering
  \includegraphics[height=7.5cm]{figures/a2cris_coef_LW.pdf}
  \caption{LW $a$ and $b$ weights for the linear correction $ax+b$}
  \label{coefLW}
\end{figure}

\begin{figure} % source a2cris_regr2.m
  \centering
  \includegraphics[height=7.5cm]{figures/a2cris_regr_MW.pdf}
  \caption{Mean and standard deviation of MW corrected apodized
    residuals.}
  \label{statMW}
\end{figure}

\begin{figure} % source a2cris_regr2.m
  \centering
  \includegraphics[height=7.5cm]{figures/a2cris_regr_SW.pdf}
  \caption{Mean and standard deviation of SW corrected apodized
    residuals.}
  \label{statSW}
\end{figure}

\begin{figure} % source a2cris_test2.m
  \centering
  \includegraphics[height=7.5cm]{figures/a2cris_regr_all.pdf}
  \caption{Mean corrected apodized residuals for all three bands.}
  \label{statall}
\end{figure}

\begin{figure} % source nedn_test1.m
  \centering
  \includegraphics[height=7.5cm]{figures/a2cris_nedn.pdf}
  \caption{{\airs} to {\cris} unapodized and apodized NEdN
    estimates}
  \label{nedn}
\end{figure}

We can give a reasonable estimate of noise equivalent differential
radiance (NEdN) as follows.  We start with {\airs} L1c and {\cris}
NEdN estimates.  The L1c spec is the average over a day (4 Dec 2016)
of NEdN values from the L1c granules, with gaps for the synthetic
channels filled in by interpolation.  The {\cris} values are from a
single {\ccast} \ref{ccast} granule.  The {\cris} values are quite
stable over time and consistent with other {\cris} NEdN estimates.
Results are summarized in figure \ref{nedn}.

Noise with a normal distribution at the {\airs} spec is added to
true {\airs} and then measured.  The measured {\airs} noise is the
``test'' line in the plot.  This is very close to the spec and so
serves as a sanity check for our methods.  True {\airs} with added
noise is then translated to {\cris} and the noise of the translation
is measured.  This is the {\airs} to {\cris} line in the plots.  The
unapodized translation tracks the {\airs} noise spec fairly closely
in the LW and MW, and is a little less in the SW.  Unapodized
{\airs} to {\cris} noise is a little higher than true {\cris} noise
in the LW, a little less in the MW, and significantly less in the
SW, and this relationship is unchanged with apodization.

Translation with deconvolution works significantly better than
interpolation for the {\airs} to {\cris} translation.  We consider
two cases.  For the first, start with true {\airs} and interpolate
radiances directly to the {\cris} user grid with a cubic spline.
For the second, interpolate true {\airs} to the 0.1 {\wn}
intermediate grid with a cubic spline and then convolve this to the
use {\cris} user grid.  Figure~\ref{intpLW} shows interpolated
{\cris} minus true {\cris} for the LW band, without apodization.
The two-step interpolation works a little better than the simple
spline, but both residuals are significantly larger than for the
translation with deconvolution.  Results for the MW are similar,
while the unapodized comparison is less clear for the SW.  With
Hamming apodization, the residuals with deconvolution are
significantly less than interpolation for all three bands.

\begin{figure} % source a2cris_test1
  \centering
  \includegraphics[height=7.5cm]{figures/a2cris_interp_LW.pdf}
  \caption{spline interpolation, interpolation with convolution, 
    and deconvolution with convolution for the {\cris} LW band}
  \label{intpLW}
\end{figure}

% \FloatBarrier
% \section{Deconvolution to constant resolving power}
% \label{airsL1d}
% 
% Section \ref{decon} raised the question of the inherent resolution
% of the deconvolution.  There we compared the deconvolution (without
% reconvolution) to calculated radiances with an oversampled resolving
% power of 2000.  But the residual was quite large.  
% 
% Similar to the situation with convolution to the {\cris} user grid,
% 

\FloatBarrier
\section{Translation to an idealized grating model}
\label{airsL1d}

The {\airs} deconvolution can be used for other translations.  In
this section we briefly consider reconvolution to idealized grating
models with resolving power of 1200 and 700.  Define an {\airs} L1d
basis with the generalized Gaussian response function above, with
$\fwhm = v / \hbox{resolving power}$ and $dv = \fwhm / 2$, and with
the $dv$-spaced channel steps starting at $v_0$.  In contrast with
the regular spacing used for the {0.1~\wn} intermediate grid, this
is not oversampled.

% For this and similar comparisons both the reference truth (``true
% L1d'') and the reconvolution targer (L1c to L1d) are done using the
% same L1d channel sets--that is, sets with the same resolving power
% and starting channel.  

As for tests of the {\airs} to {\cris} translation, `true L1c'' is
calculated by convolving the kcarta radiances with {\airs} L1c SRFs
and ``true L1D'' by convolving to an L1d basis with the desired
resolving power and first channel frequency.  L1c is translated to
L1d by deconvolution followed by reconvolution to the desired L1d
basis to get ``L1C to D'', and this is compared with true L1d.

Figure \ref{L1d1200} shows residuals for reconvolution to an L1d
basis with resolving power of 1200, the nominal {\airs} resolution
and figure \ref{L1d700s} shows residuals for a resolving power of
700.  Note the different x-axes for the two figures.  Some resolving
power is sacrificed in shifting channel centers to a single regular
function of frequency.  The L1d residuals dependend in part on the
starting channel, and so on how the SRF peaks line up with the L1c
set.  The residuals above are the result of a rough fit for $v_0$.
For a resolving power of 1200 this gave $v_0$ equal to the first L1c
channel, while for 700 it was the first L1c channel plus $0.2$~\wn.

Although some resolving power is lost in the L1c to L1d translation,
the residuals for a resolving power of 1200 are roughly comparable
to unapodized {\cris}, and residuals for a resolving power of 700 to
apodized {\cris}.  As with the {\airs} to {\cris} translation,
residuals are reduced significantly with a linear correction.
Residuals for L1d with a resolving power of 700 after correction are
roughly comparable to residuals for apodized {\cris} after a similar
correction.

\begin{figure} % source L1d_regr1.m
  \centering
  \includegraphics[height=7.5cm]{figures/L1d_corr_1200.pdf}
  \caption{mean and standard deviation over the 49 fitting profiles
    for the L1c to L1d translation minus true L1d for a resolving
    power of 1200}
  \label{L1d1200}
\end{figure}

\begin{figure} % source L1d_regr1.m
  \centering
  \includegraphics[height=7.5cm]{figures/L1d_corr_700.pdf}
  \caption{mean and standard deviation over the 49 fitting profiles
    for the L1c to L1d translation minus true L1d for a resolving
    power of 700}
  \label{L1d700s}
\end{figure}

As with the {\airs} to {\cris} translation, deconvolution is
significantly better than interpolation for the L1c to L1d
translation.  We consider two cases.  For the first, start with true
L1c and interpolate radiances directly to the L1d grid with a cubic
spline.  For the second, interpolate true L1c to the 0.1 {\wn}
intermediate grid with a cubic spline and convolve this to the L1d
channel set.  Figure~\ref{interpL1d} shows interpolated L1d minus
true L1d.  The two-step interpolation works a little better than the
simple spline, but is still much larger than the residual for
translation with deconvolution.

The L1c to L1d translation can be represented as a single linear
transform $S_d\cdot S_c^{-1}$, where $S_c$ and $S_d$ are the
transforms taking the intermediate grid to L1c and L1d channels and
$S_c^{-1}$ the pseudo-inverse of $S_c$, that is, the deconvolution
transform.  We can get such a tranform in other ways, for example by
regression to find $X$ that minimizes the residual $\|X r_c -
r_d\|_2$ for L1c and L1d radiance sets $r_c$ and $r_d$.  If $r_c$ and
$r_d$ are $m$ and $n$ by $k$ matrices, then if $k <= m$ we can simply
solve for $X$.  If $k < m$ the system is underdetermined; in this
case the residual is zero but extrapolation behavior is typically
poor.  If $k > m$ we can find $X$ by regression, and extrapolation
behavior can be quite good if we regress against large sets of
representative data.  In practice this seems to work very well, at
least for minimizing both dependent and independent set residuals.
But in contrast with the sharply banded composite transform $S_d\cdot
S_c^{-1}$ the resulting transform is full of unexpected correlations
and so may not be suitable for applications where we want to trace
channel dependencies in the translation.

\begin{figure} % source L1d_test2.m
  \centering
  \includegraphics[height=7.5cm]{figures/CtoD_interp_diff.pdf}
  \caption{spline interpolation, interpolation with convolution, 
    and deconvolution with convolution for the {\airs} L1c to L1d
    translation with $v_0=649.822$~\wn\ and a resolving power of 700}
  \label{interpL1d}
\end{figure}

\FloatBarrier
\section{Direct regression}
\label{dregr}

The corrections of section \ref{airs2cris} and \ref{airsL1d} are
applied independently to each channel, while in section \ref{dregr}
we do the translation, including channel shifts, with a single
larger regression.

The L1c to L1d translation can be represented as a single linear
transform $S_d\cdot S_c^{-1}$, where $S_c$ and $S_d$ are the
transforms taking the intermediate grid to L1c and L1d channels and
$S_c^{-1}$ the pseudo-inverse of $S_c$, that is, the deconvolution
transform.  We can get such a tranform in other ways, for example by
regression to find $X$ that minimizes the residual $\|X r_c -
r_d\|_2$ for L1c and L1d radiance sets $r_c$ and $r_d$.  If $r_c$ and
$r_d$ are $m$ and $n$ by $k$ matrices, then if $k <= m$ we can simply
solve for $X$.  If $k < m$ the system is underdetermined; in this
case the residual is zero but extrapolation behavior is typically
poor.  If $k > m$ we can find $X$ by regression, and extrapolation
behavior can be quite good if we regress against large sets of
representative data.  In practice this seems to work very well, at
least for minimizing both dependent and independent set residuals.
But in contrast with the sharply banded composite transform $S_d\cdot
S_c^{-1}$ the resulting transform is full of unexpected correlations
and so may not be suitable for applications where we want to trace
channel dependencies in the translation.

\FloatBarrier
\section{Applications and conclusions}
\label{appcon}

[some notes on SNOs from SNO paper]

\FloatBarrier
\section{Appendix}
\label{append}

We can use the basis size needed to get a reconstruction residual
below some fixed threshold as a measure the correlation of a set of
radiances.  Let $R$ be an $m$ by $n$ array of radiances with one row
per channel and one column per observation.  Let $R=U S\,V^T$ be a
singular value decomposition with the singular values in descending
order, and $U_k$ the first $k$ columns of $U$.  We can simply take
the singular values as one measure of correlation; these drop off
more quickly for more correlated data.

For our applications a more useful measure is the number of basis
vectors needed for a particular reconstruction residual.  We can
approximate $R$ with a linear combination of the orthonormal vectors
$U_k$, as $R\approx U_k U_k^T R$.  The approximation improves as
$k$ increases and becomes exact for some $k <= m$.  This is useful
as a form of compression when $k$ is small relative to $n$.  For
that case we save $U_k$ and $U_k^T R$ separately.  Examples of this
include IASI radiance data and the kcarta absorption database.

We can think of $k$ as a reconstruction score and for example find
the smallest $k$ such that $\rms(B^{-1}(R) - B^{-1}(U_k U_k^T R)) <
0.02$ degrees K, where $B^{-1}$ is the inverse Planck function.  In
this case $k$ is the number of linearly independent vectors needed
to represent the radiance data to some particular accuracy, and so a
measure of the correlation of the data.  For the 49 profile fitting
set and a residual limit of 0.02~K, we found $k=48$, which we would
interpret as largely uncorrelated while for the 7377 profile cloudy
set we found $k=260$, which we would interpret as moderately
correlated.

\FloatBarrier
\bibliographystyle{abbrv}
\bibliography{decon}

\end{document}

