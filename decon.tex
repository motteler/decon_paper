\documentclass[12pt]{article}
\usepackage{graphicx}
\usepackage{placeins}

\input{crisdefs.tex}

\title{AIRS Deconvolution and Translation \\
  from the AIRS to CrIS IR Sounders \\
  \vspace{3mm}
  {****} DRAFT {****}\\
}

\author{Howard E.~Motteler \\
  L.~Larrabee Strow \\
  \\
  UMBC Atmospheric Spectroscopy Lab \\
  Joint Center for Earth Systems Technology \\
}

\date{\today}
\begin{document}
\maketitle

\section{Introduction}

Upwelling infrared radiation as measured by the {\airs} \cite{airs1}
and {\cris} \cite{cris1,cris2} sounders is a significant part of the
long term climate record.  We would like to treat this information as
a single data set but the instruments have different spectral
resolutions, channel response functions, and band spans.  As a step
in addressing this problem we consider the translation of channel
radiances from {\airs} to standard resolution {\cris}.

Translation from {\airs} to {\cris} involves more that simple
resampling for two reasons.  First, {\airs} is a grating spectrometer
with a distinct response function for each channel determined by the
focal plane geometery, while {\cris} is a Michaelson interferometer
with a sinc ILS after calibration and corrections.  Second, we can
take advantage of our detailed knowledge of the {\airs} spectral
response functions (SRFs) and their overlap to deconvolve channel
radiances to a resolution-enhanced intermediate representation.

The {\airs} to {\cris} translation then consists of two steps,
deconvolution of the {\airs} channel radiances to an intermediate
grid, typically $0.1$~\wn, the approximate resolution of the
tabulated {\airs} SRFs, followed by reconvolution to the {\cris} 
user grid.  In section \ref{airs2cris} the reconvolution as more
conventional resampling, together with channel intersection and
bandpass filtering.  In section \ref{statfix} we show how to further
improve residuals by adding a statistically based correction.

Translations are validated by comparison with calculated reference
truth.  To test the {\airs} to {\cris} translation we start with
profiles spanning a significant range of atmospheric conditions.
Upwelling radiance is calculated at a 0.0025 {\wn} grid with kcarta
\cite{kcarta1} over a band spanning the {\airs} and {\cris} response
functions.  ``True {\airs}'' is calculated from this by convolving
the kcarta radiances with the tabulated {\airs} SRFs, and ``true
{\cris}'' by convolving kcarta radiances to the {\cris} instrument
specifications.  {\airs} is then translated to {\cris} to get
``{\airs} {\cris}'' and compared with true {\cris}.  Details are
presented in sections \ref{airs2cris} and \ref{statfix}

\FloatBarrier
\section{AIRS Deconvolution}
\label{decon}

The {\airs} spectral response functions model channel response as a
function of frequency and associate channels with nominal center
frequencies.  Each {\airs} channel $i$ has an associated spectral
response function or {\srf} $\sigma_i(v)$ such that the channel
radiance $c_i = \int \sigma_i(v)r(v)\,dv$, where $r$ is radiance at
frequency $v$.  The center or peak of $\sigma_i$ is the nominal
channel frequency.

\begin{figure} % source plot_SRFs.m
  \centering
  \includegraphics[height=8cm]{figures/airs_sample_SRFs.pdf}
  \caption{sample adjacent {\airs} spectral response functions}
  \label{srfs1}
\end{figure}

\begin{figure} % source cris_test7.m
  \centering
  \includegraphics[height=8cm]{figures/airs_decon_res.pdf}
  \caption{detail of deconvolved {\airs} and kcarta 0.0025 {\wn}
    radiances convolved to a sinc ILS at 0.2 {\wn}}
  \label{dsinc}
\end{figure}

\begin{figure} % source plot_SRFs.m
  \centering
  \includegraphics[height=8cm]{figures/airs_decon_basis.pdf}
  \caption{sample basis function for the deconvolved {\airs}
    radiances}
  \label{dbasis}
\end{figure}

Figure \ref{srfs1} shows a typical subset of {\airs} SRFs.  Note the
significant overlap in the wings.  This allows the deconvolution to
recover resolution beyond that of the response functions considered
individually.  The spacing of the AIRS L1b channels is not regular;
there are both gaps and close neighbors, side effects of the focal
plane geometry.  Both the gaps and close neighbors cause problems for
a deconvolution.  The AIRS L1c channel set \cite{airs1c} is a derived
product of the 1b set with filled gaps and relatively regular (though
still frequency dependent) frequency spacing, and we will use the 1c
set here.

Suppose we have $n$ channels and a frequency grid $\vec v$ of $k$
points spanning the domains of the functions $\sigma_i$.  The grid
step size for our applications is often 0.0025 {\wn}, the kcarta
resolution.  Let $S_k$ be an $n\times k$ array such that $s_{i,j} =
\sigma_i(v_j)/w_i$, where $w_i = \sum_j \sigma_i(v_j)$, that is
where row $i$ is $\sigma_i(v)$ tabulated at the grid $\vec v$ and
normalized so the row sum is 1.  If the channel centers are in
increasing order $S_k$ is banded, and if they are not too close the
rows are linearly independent.  $S_k$ is a linear transform whose
domain is radiance at the grid $\vec v$ and whose range is channel
radiances.  If $r$ is radiance at the grid $\vec v$, then $c = S_k r$
gives a good approximation of the channel radiances $c_i = 
\int\sigma_i(v)r(v)\,dv$.

In practice this is how we convolve kcarta or other simulated
radiances to get {\airs} channel radiances.  We construct $S_k$
either explicitly or implicitly from {\airs} {\srf} tabulations.
The matrix $S_k$ in the former case is large but manageable with a
banded or sparse representation.

Suppose we have $S_k$ and channel radiances $c$ and want to 
find $r$, that is, to deconvolve $c$.  Consider the linear system
$S_k x = c$.  Since $n < k$ for the kcarta grid mentioned above this
is underdetermined, with infinitely many solutions.  We could add
constraints, take a pseudo-inverse, consider a new matrix $S_b$ with
columns tabulated at some coarser grid, or some combination of the
above.

For an {\airs} to {\cris} translation we are mainly interested in 
the transform $S_b$ with {\srf}s at an intermediate grid, typically
0.1 {\wn}, the approximate resolution of the {\srf} measurements.
Let $\vec v_b = v_1,v_2,\ldots,v_m$ be a 0.1 {\wn} grid spanning the
domains of the functions $\sigma_i$.  Similar to $S_k$, let $S_b$ be
an $n\times m$ array where row $i$ is $\sigma_i(v)$ tabulated at the
$\vec v_b$ grid, with rows normalized to~1.  If $r$ is radiance at
the $\vec v_b$ grid, then $c = S_b r$ is still a reasonable
approximation of $\int\sigma_i(v)r(v)\,dv$.

Consider the linear system $S_b x = c$, similar to the case 
$S_k x = c$ above, where we are given $S_b$ and channel signals $c$
and want to find radiances $x$.  Since $n < m < k$, as with $S_k$
the system will be underdetermined but more manageable because $m$
is approximately 40 times less than $k$.  We use a Moore-Penrose
pseudoinverse as $S_b^{-1}$.  Then $x = S_b^{-1} c$ gives us
deconvolved radiances at the {\srf} tabulation grid. 

The {\airs} deconvolution gives a significant resolution enhancement.
Figure \ref{dsinc} shows LW detail of deconvolved {\airs} together
with kcarta radiances convolved directly to a 0.2 {\wn} sinc ILS.
Figure \ref{dbasis} shows a typical basis function for the {\airs}
deconvolution, that is, a column of the pseudo-inverse $S_b^{-1}$.

\FloatBarrier
\section{AIRS to CrIS translation}
\label{airs2cris}

For the {\cris} standard resolution mode the channel spacing is
$0.625$ {\wn} for the LW, 1.25 {\wn} for the MW, and 2.5 {\wn} for
the SW bands.  The first step in the {\airs} L1c to {\cris}
translation is to deconvolve the {\airs} channel radiances to a 0.1
{\wn} intermediate grid, the nominal {\airs} SRF resolution.  Then
for each {\cris} band,

\begin{itemize}
  \item find the {\airs} and {\cris} band intersection

  \item apply a bandpass filter to the deconvolved {\airs} radiances
    to restrict them to the intersection, with a rolloff outside the
    intersection

  \item reconvolve the filtered spectra to the {\cris} user grid

\end{itemize}

Translations are validated by comparison with calculated reference
truth.  For the results presented in this section we start with 49
fitting profiles spanning a significant range of atmospheric
conditions \cite{sarta1,sarta2}.  Upwelling radiance is calculated at
a 0.0025 {\wn} grid with kcarta \cite{kcarta1} over a band spanning
the {\airs} and {\cris} response functions.  ``True {\airs}'' is
calculated from this by convolving the kcarta radiances with {\airs}
SRFs, and ``true {\cris}'' by convolving kcarta radiances to the
{\cris} instrument specifications.  {\airs} is then translated to
{\cris} (we call this ``{\airs} {\cris}'') and compared with true
{\cris}.  This sort of validation assumes perfect knowledge of the
{\airs} and {\cris} instrument response functions and so gives only a
lower bound on residuals, and on how well the translations can work
in practice.  The better we know the response functions, the closer
practical translations can approach these limits.

Figure \ref{aclws} shows true {\cris}, true {\airs}, deconvolved
{\airs}, and {\airs} {\cris}.  In the first subplot we mainly see the
greater fine structure in the deconvolution.  The second subplot
shows details from 660 to 680 {\wn}.  The remaining figures show true
{\cris} minus {\airs} {\cris} for the 49 fitting profiles, with and
without Hamming apodization, for each of the {\cris} bands.  The
residuals are significantly reduced with apodization.

\begin{figure} % source a2cris_test1
  \centering
  \includegraphics[height=8cm]{figures/a2cris_spec_LW.pdf}
  \caption{true {\cris}, true {\airs}, deconvolved {\airs}, and
    {\airs} {\cris}}
  \label{aclws}
\end{figure}

\begin{figure} % source a2cris_test1
  \centering
  \includegraphics[height=8cm]{figures/a2cris_diff_LW.pdf}
  \caption{Mean and standard deviation of unapodized and Hamming
    apodized {\airs} {\cris} minus true {\cris}, for the {\cris} LW
    band}
  \label{aclwd}
\end{figure}

\begin{figure} % source a2cris_test1
  \centering
  \includegraphics[height=8cm]{figures/a2cris_diff_MW.pdf}
  \caption{Mean and standard deviation of unapodized and Hamming
    apodized {\airs} {\cris} minus true {\cris}, for the {\cris} MW
    band}
  \label{aclwd}
\end{figure}

\begin{figure} % source a2cris_test1
  \centering
  \includegraphics[height=8cm]{figures/a2cris_diff_SW.pdf}
  \caption{Mean and standard deviation of unapodized and Hamming
    apodized {\airs} {\cris} minus true {\cris}, for the {\cris} SW
    band}
  \label{aclwd}
\end{figure}

\begin{figure} % source a2cris_test1
  \centering
  \includegraphics[height=8cm]{figures/a2cris_interp_LW.pdf}
  \caption{spline interpolation, interpolation with convolution, 
    and deconvolution with convolution for the {\cris} LW band}
  \label{intpLW}
\end{figure}

Deconvolution works better for the {\airs} to {\cris} 
translation than either interpolation or interpolation (rather than
deconvolution) to an intermediate grid followed by convolution to
{\cris} radiances.  For the first case we start with true {\airs} and
interpolate radiances directly to the {\cris} user grid with a cubic
spline.  For the second we interpolate true {\airs} to the 0.1 {\wn}
intermediate grid with a cubic spline and then convolve this to the
use {\cris} user grid.

Figure \ref{intpLW} shows interpolated {\cris} minus true {\cris} 
for the LW band, without any apodization.  While the two-step
interpolation works a little better than the simple spline, both
residuals are significantly larger than for the translation with
deconvolution.  Results for the MW and SW are similar.  Deconvolution
is significantly better for the MW, while the comparison is less
clear for the SW.  Comparisons with Hamming apodization show the
residuals with deconvolution are significantly less for all three
bands.

\FloatBarrier
\section{Statisitcal Refinement}
\label{statfix}

\FloatBarrier
\bibliographystyle{abbrv}
\bibliography{decon}

\end{document}

