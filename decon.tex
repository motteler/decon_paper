\documentclass[11pt]{article}
\usepackage{graphicx}
\usepackage{placeins}

\input{crisdefs.tex}

\title{AIRS Deconvolution and Translation \\
  from the AIRS to CrIS IR Sounders \\
  \vspace{3mm}
  {****} DRAFT {****}\\
}

\author{Howard E.~Motteler \\
  L.~Larrabee Strow \\
  \\
  UMBC Atmospheric Spectroscopy Lab \\
  Joint Center for Earth Systems Technology \\
}

\date{\today}
\begin{document}
\maketitle

\section{Introduction}

Upwelling infrared radiation as measured by the {\airs} \cite{airs1}
and {\cris} \cite{cris1,cris2} sounders is a significant part of the
long term climate record.  We would like to treat this as a single
data set and often want to compare radiances, for example in the
analysis of simultaneous nadir overpasses (SNOs) for sounder
calibration or validation.  However the instruments have different
spectral resolutions, channel response functions, and band spans.
As a step in addressing this problem we consider the translation of
channel radiances from {\airs} to standard resolution {\cris}.

In addition to {\airs} to {\cris} we make regular use of an {\iasi}
to {\cris} translation, and have implemented and tested {\iasi} to
{\airs} and {\cris} to {\airs} translations as well.  The
translations from {\iasi} includes deapodization (a form of
deconvolution) before reconvolution to the translation target and
work very well.  Ranking these translations by accuracy in
comparison with calculated reference truth, we have {\iasi} to
{\cris}, {\iasi} to {\airs}, {\airs} to {\cris}, and finally {\cris}
to {\airs} \cite{deconATBD}.  But aside from the {\airs} to {\cris}
translation, the methods used are for the most part conventional.

Our translation from {\airs} to {\cris} has some novel features.
{\airs} is a grating spectrometer with a distinct response function
for each channel determined by the focal plane geometery, while
{\cris} is a Michaelson interferometer with a sinc response
function, after calibration and corrections.  In section \ref{decon}
we show how to take advantage of our detailed knowledge of the
{\airs} spectral response functions (SRFs) and their overlap to
deconvolve channel radiances to a resolution-enhanced intermediate
representation, typically a $0.1$~\wn\ grid, the approximate
resolution of the tabulated {\airs} SRFs.  
This intermediate representation can then be reconvolved to an
alternate instrument specification.  Section \ref{airs2cris} gives
details and validation tests for the {\airs} to {\cris} translation
and section \ref{airsL1d} for translation to an idealized grating
model.  For both cases deconvolution followed by reconvolution is
shown to work significantly better than conventional interpolation.
Both methods can be further improved with a statistical correcion.
In section \ref{dregr} we consider a purely statistical approach to
such translations.  Section \ref{appcon} discusses applications and
related and future work.

%---------------------------------------------------------------------
\FloatBarrier
\section{AIRS Deconvolution}
\label{decon}

The {\airs} spectral response functions model channel response as a
function of frequency and associate channels with nominal center
frequencies.  Each {\airs} channel $i$ has an associated spectral
response function or {\srf} $\sigma_i(v)$ such that the channel
radiance $c_i = \int \sigma_i(v)r(v)\,dv$, where $r$ is radiance at
frequency $v$.  The center or peak of $\sigma_i$ is the nominal
channel frequency.

\begin{figure} % source plot_SRF2.m
  \centering
  \includegraphics[height=7.5cm]{figures/airs_sample_srfs.pdf}
  \caption{sample {\airs} spectral response functions from the low
    and high ends of the band.   The dashed line is a generalized
    Gaussian function.}
  \label{srfs1}
\end{figure}

\begin{figure} % source plot_SRF2.m
  \centering
  \includegraphics[height=7.5cm]{figures/airs_L1c_res.pdf}
  \caption{{\airs} L1c channel spacing and derived resolving
    power.}
  \label{chan1}
\end{figure}

Figure \ref{srfs1} shows typical {\airs} SRFs from the low and high
ends of the band.  Note the significant overlap in the wings.  This
can allow for a deconvolution to recover resolution beyond that of
the response functions considered individually.  The SRFs are not
necessarily symmetrical, especially at the high end of the band.
The dashed line on top of the third SRF in each group is a fit for a
generalized Gaussian, which we consider in more detail later in this
section.  Figure \ref{chan1} shows channel spacing and resolving
power for the {\airs} L1c channel set \cite{airs1c}.  The variable
channel spacing and resolving power are due to the modular structure
of the focal plane.  Although not entirely regular---that is, not a
simple function of frequency---the L1c channel set is more regular
than the L1b channel set from which it is derived, and we mainly
consider the L1c set here.

Suppose we have $n$ channels and a frequency grid $\vec v$ of 
$k$ points spanning the union of the domains of the functions
$\sigma_i$.  The grid step size for our applications is often 0.0025
{\wn}, the default resolution for upwelling radiances calculated
with kcarta \cite{kcarta1}.  Let $S_k$ be an $n\times k$ array such 
that $s_{i,j} = \sigma_i(v_j)/w_i$, where $w_i = \sum_j \sigma_i(v_j)$,
that is where row $i$ is $\sigma_i(v)$ tabulated at the grid $\vec
v$ and normalized so the row sum is 1.  If the channel centers are
in increasing order $S_k$ is banded, and if they are not too close
(as is the case for a few of the L1b channels) the rows are linearly
independent.  $S_k$ is a linear transform whose domain is radiance
at the grid $\vec v$ and whose range is channel radiances.  If $r$
is radiance at the grid $\vec v$, then $c = S_k r$ gives a good
approximation of the channel radiances $c_i = \int\sigma_i(v)r(v)\,dv$.
In practice this is how we convolve kcarta or other high resolution
calculated radiances to get {\airs} channel radiances, for example
for reference truth or ``true {\airs}'' for the tests shown here.

% We construct $S_k$ either explicitly or implicitly from the
% {\airs} {\srf} tabulations.  The matrix $S_k$ in the former case
% is large but manageable with a banded or sparse representation.

For the {\airs} to {\cris} and other translations we are mainly
interested in the transform $S_b$ for {\srf}s at an intermediate
resolution, typically $0.1~\wn$.  This is the approximate resolution
of the {\srf} measurements and convenient for reconvolution to the
{\cris} user grid.  So let $\vec v_b = v_1,v_2,\ldots,v_m$ be a
$0.1~\wn$ grid spanning the domains of the functions $\sigma_i$.
Similar to $S_k$, let $S_b$ be an $n\times m$ array where row $i$ is
$\sigma_i(v)$ tabulated at the $\vec v_b$ grid, with rows normalized
to~1.  If $r$ is radiance at the $\vec v_b$ grid, then $c = S_b r$
is still a reasonable approximation of $\int\sigma_i(v)r(v)\,dv$.

For our application we want to start with $c$ and find $r$, that is
to deconvolve $c$ by solving $S_b r = c$ for $r$.  Since $m < k$ the
system is underdetermined.  We take the Moore-Penrose pseudoinverse
\cite{pinv} of $S_b$ to get $r_0 = S_b^{-1} c$.  This gives a
minimal solution, in the sense that $||r_0||_2 \le ||r_j||_2$ for
all $r_j$ satisfying $S_b r_j = c$.  The condition number for $S_b$
as built from the L1c channels is $||S_b||_2||S_b^{-1}||_2 = 115$,
which is tolerable.

Although our main goal is to reconvolve the $0.1~\wn$ intermediate
representation to the {\cris} or other user grids, we first compare
the deconvolved radiances with reference truth from a direct
convolution to the intermediate grid.  The choice of response
functions for the direct convolution is not obvious, since the
deconvolution is undoing---at least to some extent---the effects of
the {\airs} SRF convolutions.  We chose a generalized Gaussian of
the form
\[w(v, v_0, \fwhm) = 
\exp\left(-\left(\frac{(v - v_0)^2}{2c^2}\right)^{1.5}\right) \]
where $c=\fwhm / (2\sqrt{2\ln 2})$ and $v_0$ is the desired channel
center.  The exponent $1.5$ was chosen to give an approximate match
to {\airs} SRFs with the same {\FWHM} and channel centers, though
without the fine structure and variation of the measured \srf s.
Figure \ref{srfs1} shows two such generalized Gaussians paired with
the corresponding {\airs} SRFs.  We used the same functions as
reference truth for the $0.1~\wn$ intermediate grid with ${\fwhm} =
v_i / 2000$, where $v_i$ are the grid frequencies.  This represents
a hypothetical grating spectrometer with a resolving power of 2000,
oversampled to the 0.1~\wn\ grid.  The residual was roughly
minimized for a resolving power of 2000, as shown here.  We also
tried the generalized Gaussian with a fixed {\FWHM} for values
$0.4$, $0.6$, and $0.8$, and a sinc basis with a spacing of
$0.2$~\wn, all of which gave larger residuals.

\begin{figure} % source decon_test1.m 
  \centering
  \includegraphics[height=7.5cm]{figures/airs_decon_zoom.pdf}
  \caption{details from fitting profile 1 for kcarta, direct
    convolution to the $0.1$~\wn\ grid (``gauss''), deconvolved
    {\airs}, and true {\airs}.}
  \label{dzoom}
\end{figure}

\begin{figure} % source decon_test1.m
  \centering
  \includegraphics[height=7.5cm]{figures/airs_decon_spec.pdf}
  \caption{spectra from fitting profile 1 for direct convolution to
    the $0.1$~\wn\ grid (``gauss'') and deconvolved {\airs}}
  \label{dspec}
\end{figure}

% \begin{figure} % source decon_test1.m
%   \centering
%   \includegraphics[height=7.5cm]{figures/airs_decon_diff.pdf}
%   \caption{mean and standard deviation over the 49 fitting profiles
%     for the L1c deconvolution minus direct convolution to the
%     $0.1$~\wn\ intermediate grid.  The residuals are too large to use
%     the deconvolved radiances directly.}
%   \label{ddiff}
% \end{figure}

The {\airs} deconvolution gives a modest resolution enhancement, at
the cost of added artifacts and noise.  Figure \ref{dzoom} shows
details of kcarta, direct convolution to the $0.1$~\wn\ grid
(``gauss''), deconvolution, and AIRS spectra for fitting profile~1
\cite{sarta1,sarta2}.  In the first subplot we see the deconvolution
is capturing some of the fine structure in the kcarta data that is
present in the direct convolution but not in the AIRS data.  In the
second subplot we see the deconvolution (and direct convolution)
resolving a pair of close lines that are not resolved at the {\airs}
L1c resolution.  But we also see some ringing that is not present in
the direct convolution.  Figure \ref{dspec} shows the full spectra
from fitting profile~1, along with sample details from the low and
high ends of the band, for the deconvolution and direct convolution
to the intermediate grid.  In the details we see some overshoot and
ringing in the deconvolution.  But as noted we do not propose using
the deconvolved radiances directly, they are an intermediate step in
reconvolution to a lower resolution.

% Figure \ref{ddiff} shows the mean and standard deviation of the
% difference of the deconvolved minus the directly convolved radiances
% for all 49 fitting profiles.  The residuals are large but mainly
% significant for understanding limitations of the deconvolution.

% The residuals can be reduced dramatically by reconvolving the
% $0.1$~\wn\ intermediate grid to a lower resolution.  We consider
% this for convolution to the {\cris} user grid in the next section.

\begin{figure} % source plot_Binv.m
  \centering
  \includegraphics[height=7.5cm]{figures/airs_decon_basis.pdf}
  \caption{sample adjacent rows for the deconvolution and L1c to L1d
    transforms}
  \label{dbasis}
\end{figure}

Figure \ref{dbasis} shows a pair of typical adjacent rows of the
deconvolution matrix $S_b^{-1}$\, in the first subplot.  Row $i$ of
$S_b^{-1}$ is the weights applied to L1c channel radiances to
synthesize the deconvolved radiance $r_i$ at the intermediate grid
frequency $v_i$.  The oscillation shows we are taking the closest
AIRS channel, subtracting weighted values for channels $\pm 1$ step
away, adding weighted values for channels $\pm 2$ steps away, and so
on, with the weights decreasing quickly as we move away from $v_i$,
with eight to ten L1c channels making a significant contribution to
each deconvolution grid point.

The second subplot shows four adjacent rows of the matrix 
$S_d \cdot S_b^{-1}$, which takes L1c to L1d channel radiances.
(The L1d radiances are discussed in a later section; here they are
of interest mainly as a typical reconvolution.)  Both matrices are
banded but the bands are narrower in the second, with three to five
L1c channels contributing significantly to each L1d channel.  
The range of influence is significant since for example we may want
to see which L1d channels are derived in part from the subset of
synthetic L1c channels.

%---------------------------------------------------------------------
\FloatBarrier
\section{AIRS to CrIS translation}
\label{airs2cris}

Given {\airs} deconvolution to a $0.1~\wn$ intermediate grid,
reconvolution to the {\cris} user grid is straightforward.  For the
{\cris} standard resolution mode the channel spacing is $0.625~\wn$
for the LW, $1.25~\wn$ for the MW, and $2.5~\wn$ for the SW bands.
For each {\cris} band, we (1) find the {\airs} and {\cris} band
intersection, (2) apply a bandpass filter to the deconvolved {\airs}
radiances restricting them to the intersection, with a rolloff
outside the intersection, and (3) reconvolve the filtered spectra to
the {\cris} user grid with a zero-filled double Fourier transform
\cite{motfft}. 

Translations are tested by comparison with calculated reference
truth.  We start with a set of atmospheric profiles and calculate
upwelling radiance at a $0.0025~\wn$ grid with kcarta \cite{kcarta1}
over a band spanning the domains of the {\airs} and {\cris} response
functions.  ``True {\airs}'' is calculated by convolving the kcarta
radiances with {\airs} SRFs and ``true {\cris}'' by convolving
kcarta radiances to a sinc basis at the {\cris} user-grid.  True
{\airs} is then translated to {\cris} to get ``{\airs} {\cris}'',
and this is compared with true {\cris}.  Figure~\ref{specLW} shows
sample spectra for true {\airs}, deconvolved {\airs}, true {\cris}
and {\airs} {\cris}.  The difference between true {\cris} and
{\airs} {\cris} is hard to see at this level of detail, and for the
remainder of this paper we will mainly show explicit differences.

For most tests we use a set of 49 fitting profiles spanning a wide
range of clear atmospheric conditions, initially chosen for testing
radiative transfer codes \cite{sarta1,sarta2}.  The set is largely
uncorrelated; reducing the reconstruction residual to 0.02~K
requires 48 left-singular vectors.  (Details of this correlation
measure are given in an appendix.)  For statistical correction and
the direct regression discussed in section \ref{dregr} we also use a
set of 7377 radiances calculated from mostly cloudy AIRS profiles
spanning several consecutive days as the dependent set.  This set is
moderately correlated; reducing the reconstruction residual to
0.02~K requires 260 left-singular vectors.  Splitting the 7377
profile set into dependent and independent subsets and comparing
residuals from the independent subset with residuals from the
49-profile set, we found residuals from the latter are consistently
larger, suggesting it makes for a stricter test.  So for the results
shown here the test or independent set is always the 49-profile set,
while for tests requiring fitting the 7377 profile is used as the
dependent set.

Figures \ref{diffLW}, \ref{diffMW}, and \ref{diffSW} show the mean
and standard deviation of true {\cris} minus {\airs} {\cris} for the
49 fitting profiles, with and without Hamming apodization, for each
of the {\cris} bands.  Figure \ref{meanAll} summarizes these results
for Hamming apodized radiances.  The residual has a high frequency
component with a period of 2 channel steps that is significantly
reduced by the apodization.  The constant or DC bias is very close
to zero for the apodized residuals.

% $0.002$~K for the LW, $-0.005$~K for the MW, and $0.001$~K for the
% SW.

\begin{figure} % source a2cris_test1
  \centering
  \includegraphics[height=7.5cm]{figures/a2cris_spec_LW.pdf}
  \caption{true {\airs}, deconvolved {\airs}, true {\cris}, and
    {\airs} {\cris}}
  \label{specLW}
\end{figure}

\begin{figure} % source a2cris_test1
  \centering
  \includegraphics[height=7.5cm]{figures/a2cris_diff_LW.pdf}
  \caption{Mean and standard deviation of unapodized and Hamming
    apodized {\airs} {\cris} minus true {\cris}, for the {\cris} LW
    band}
  \label{diffLW}
\end{figure}

\begin{figure} % source a2cris_test1
  \centering
  \includegraphics[height=7.5cm]{figures/a2cris_diff_MW.pdf}
  \caption{Mean and standard deviation of unapodized and Hamming
    apodized {\airs} {\cris} minus true {\cris}, for the {\cris} MW
    band}
  \label{diffMW}
\end{figure}

\begin{figure} % source a2cris_test1
  \centering
  \includegraphics[height=7.5cm]{figures/a2cris_diff_SW.pdf}
  \caption{Mean and standard deviation of unapodized and Hamming
    apodized {\airs} {\cris} minus true {\cris}, for the {\cris} SW
    band}
  \label{diffSW}
\end{figure}

\begin{figure} % source a2cris_test1
  \centering
  \includegraphics[height=7.5cm]{figures/a2cris_diff_all.pdf}
  \caption{Mean of apodized residuals for all three {\cris} bands}
  \label{meanAll}
\end{figure}

Some regularity remains in the apodized residual, including the
oscillation with a period of two channel steps.  Up to this point
there as been no statistical component to our translation, beyond
the choice of test set for validation.  We feel it is important to
be clear about any steps that require statistical fitting.  That
said, a simple linear correction can give a significantly further
reduction of the residuals.  For such tests as noted we use the set
of 7377 mostly cloudy AIRS profiles as the dependent set and the 49
profile set the independent or test set.

We compare three such corrections.  These are done with a separate
regression for each {\cris} channel, and so introduce no
cross-correlations.  Let $\Ttc_i$ be true {\cris} and $\Tac_i$
{\airs} to {\cris} brightness temperatures for {\cris} channel $i$,
from the dependent set.  For the bias test we subtract the mean
residual from the dependent set.  For the linear test we find $a_i$
and $b_i$ to minimize $||a_i\,\Tac_i + b_i - \Ttc_i||_2$, and for
the quadratic test weights $c_i$, $a_i$ and $b_i$ to minimize
$||c_i\,(\Tac_i)^2 + a_i\,\Tac_i + b_i - \Ttc_i||_2$.  The resulting
correction is then applied to the independent set, the 49 fitting
profiles, for comparison with true {\cris}.

Figure \ref{statLW} is a comparison of bias, linear, and quadratic
corrections for the LW band.  The linear and quadratic corrections
are nearly identical, and the quadratic coefficient is very close to
zero.  Figure \ref{coefLW} shows the weights for the linear fits
from figure \ref{statLW}.  The $a$ weights are very close to 1 and
the $b$ weight to the bias.  Figures \ref{statMW} and \ref{statSW}
show the linear correction giving a similar improvement in the MW
and a small improvement in the SW, where the quadratic correction is
noticably worse.  Figure \ref{statall} summarizes results for the
linear correction, paired with the apodized uncorrected residuals.

% Note that convolution, deconvolution, and apodization are done
% with radiances while spectra are presented and statistics done
% after translation to brightness temperatures.

\begin{figure} % source a2cris_regr2.m
  \centering
  \includegraphics[height=7.5cm]{figures/a2cris_regr_LW.pdf}
  \caption{Mean and standard deviation of LW corrected apodized
    residuals}
  \label{statLW}
\end{figure}

\begin{figure} % source a2cris_regr2.m
  \centering
  \includegraphics[height=7.5cm]{figures/a2cris_coef_LW.pdf}
  \caption{LW $a$ and $b$ weights for the linear correction $ax+b$}
  \label{coefLW}
\end{figure}

\begin{figure} % source a2cris_regr2.m
  \centering
  \includegraphics[height=7.5cm]{figures/a2cris_regr_MW.pdf}
  \caption{Mean and standard deviation of MW corrected apodized
    residuals.}
  \label{statMW}
\end{figure}

\begin{figure} % source a2cris_regr2.m
  \centering
  \includegraphics[height=7.5cm]{figures/a2cris_regr_SW.pdf}
  \caption{Mean and standard deviation of SW corrected apodized
    residuals.}
  \label{statSW}
\end{figure}

\begin{figure} % source a2cris_test2.m
  \centering
  \includegraphics[height=7.5cm]{figures/a2cris_regr_all.pdf}
  \caption{Mean corrected and uncorrected apodized residuals for all
    three bands.}
  \label{statall}
\end{figure}

\begin{figure} % source nedn_test1.m
  \centering
  \includegraphics[height=7.5cm]{figures/a2cris_nedn.pdf}
  \caption{{\airs} to {\cris} unapodized and apodized NEdN
    estimates}
  \label{nedn}
\end{figure}

We can give a reasonable estimate of noise equivalent differential
radiance (NEdN) for the translation as follows.  We start with
{\airs} L1c and {\cris} NEdN estimates.  These noise specs are
showin in figure \ref{nedn}.  The L1c spec is the average over a day
(4 Dec 2016) of NEdN values from the L1c granules, with gaps for the
synthetic channels filled by interpolation.  The {\cris} values are
from a single {\ccast} \ref{ccast} granule for the same day.  The
{\ccast} values are quite stable over time and consistent with NOAA
{\cris} NEdN estimates \cite{zav2013}.  Noise with a normal
distribution at the {\airs} spec is added to true {\airs} and then
measured.  The measured {\airs} noise is the ``test'' line in the
plot.  This is very close to the spec and so serves as a check for
our methods.  True {\airs} with added noise is then translated to
{\cris} and the noise of the translation is measured.  This is the
{\airs} to {\cris} line in the plots.  The unapodized translation
tracks the {\airs} noise spec fairly closely in the LW and MW, and
is a little less in the SW.  Unapodized {\airs} to {\cris} noise is
a little higher than true {\cris} noise in the LW, a little less in
the MW, and significantly less in the SW, and this relationship is
unchanged with apodization.

{\airs} to {\cris} translation via de- and re-convolution works
significantly better than conventional interpolation.  We consider
two cases.  For the first, start with true {\airs} and interpolate
radiances directly to the {\cris} user grid with a cubic spline.
For the second, interpolate true {\airs} to the 0.1 {\wn}
intermediate grid with a cubic spline and then convolve this to the
use {\cris} user grid.  Figure~\ref{intpLW} shows interpolated
{\cris} minus true {\cris} for the LW band, without apodization.
The two-step interpolation works a little better than the simple
spline, but both residuals are significantly larger than for the
translation with deconvolution.  Results for the MW are similar,
while the unapodized comparison is less clear for the SW.  With
Hamming apodization, the residuals with deconvolution are
significantly less than interpolation for all three bands.

\begin{figure} % source a2cris_test1
  \centering
  \includegraphics[height=7.5cm]{figures/a2cris_interp_LW.pdf}
  \caption{spline interpolation, interpolation with convolution, 
    and deconvolution with convolution for the {\cris} LW band}
  \label{intpLW}
\end{figure}

%---------------------------------------------------------------------
\FloatBarrier
\section{Translation to an idealized grating model}
\label{airsL1d}

% The nominal {\airs} resolution is 1200, though for many modules
% the real resolving power is higher.

The {\airs} deconvolution can be used for other translations.  
In this section we briefly consider reconvolution to an idealized
grating model for resolving powers of 700 and 1200.  Define an
{\airs} L1d basis with resolving power $R$ from the generalized
Gaussian response function of section \ref{decon} as follows.
Let $v_0$ be the frequency of the first channel and for $i\ge0$
$\fwhm_i = v_i / R$, $dv_i = \fwhm_i / 2$, and $v_{i+1} = v_i +
dv_i$.  As with tests of the {\airs} to {\cris} translation, true
L1c is calculated by convolving kcarta radiances with {\airs} L1c
SRFs and true L1d by convolving with an L1d basis at the desired
resolving power.  L1c is translated to L1d by deconvolution followed
by reconvolution to the desired L1d basis, and this is compared with
true L1d.

Figure \ref{L1d1200} shows residuals for reconvolution to an L1d
basis with resolving power of 1200, the nominal {\airs} resolution,
and figure \ref{L1d700s} shows residuals for a resolving power of
700.  Note the different x-axes for the two figures.  The residuals
depend in part on the L1d starting channel $v_0$, and so on how the
L1c and L1d SRF peaks line up.  The residuals shown are the result
of a rough fit for $v_0$.  For a resolving power of 1200 this gave
$v_0$ equal to the first L1c channel, while for 700 it was the first
L1c channel plus $0.2$~\wn.

We see that for both the {\airs} to {\cris} and L1c to L1d
translations some resolving power is sacrificed in shifting channel
centers to a single regular function of frequency.  Residuals for a
resolving power of 1200 (figure \ref{L1d1200}) are roughly
comparable to unapodized {\cris} (figures \ref{diffLW},
\ref{diffMW}, and \ref{diffSW}) and residuals for a resolving power
of 700 (figure \ref{L1d700s}) are roughly comparable to apodized
{\cris} (figure \ref{statall}).  As with the {\airs} to {\cris}
translation, the L1c to L1d residuals are significantly reduced with
a linear correction.  Residuals for L1d with a resolving power of
700 after correction are comparable to residuals for apodized
{\cris} after a similar correction.

\begin{figure} % source L1d_regr1.m
  \centering
  \includegraphics[height=7.5cm]{figures/L1d_corr_1200.pdf}
  \caption{mean and standard deviation over the 49 fitting profiles
    for the L1c to L1d translation minus true L1d for a resolving
    power of 1200}
  \label{L1d1200}
\end{figure}

\begin{figure} % source L1d_regr1.m
  \centering
  \includegraphics[height=7.5cm]{figures/L1d_corr_700.pdf}
  \caption{mean and standard deviation over the 49 fitting profiles
    for the L1c to L1d translation minus true L1d for a resolving
    power of 700}
  \label{L1d700s}
\end{figure}

As with the {\airs} to {\cris} translation, deconvolution works
significantly better than interpolation.  We consider similar cases.
For the first, start with true L1c and interpolate radiances
directly to the L1d grid with a cubic spline.  For the second,
interpolate true L1c to the 0.1 {\wn} intermediate grid with a cubic
spline and convolve this to the L1d channel set.
Figure~\ref{interpL1d} shows interpolated L1d minus true L1d.  The
two-step interpolation works a little better than the simple spline,
but is still much larger than the residual for translation with
deconvolution.

\begin{figure} % source L1d_test2.m
  \centering
  \includegraphics[height=7.5cm]{figures/CtoD_interp_diff.pdf}
  \caption{spline interpolation, interpolation with convolution, 
    and deconvolution with convolution for the {\airs} L1c to L1d
    translation with $v_0=649.822$~\wn\ and a resolving power of 700}
  \label{interpL1d}
\end{figure}

%---------------------------------------------------------------------
\FloatBarrier
\section{Direct and principal component regression}
\label{dregr}

The {\airs} L1c to L1d translation can be done with a single 
linear transform $S_d\cdot S_c^{-1}$, where $S_c$ and $S_d$ are the
transforms taking the intermediate grid to L1c and L1d channels.  
A similar one-step transform is possible for the {\airs} to {\cris}
translation if we use a resampling matrix rather than double Fourier
interpolation.  We can get such a tranform in other ways.  For
example if $r_a$ and $r_c$ are $m \times k$ and $n \times k$ {\airs}
and {\cris} radiance sets, we can find $X$ to minimize $\|X r_a -
r_c\|_2$.  Typically $k > m$, giving an overdetermined system, and
we solve $r_a^t X^t = r_c^t$ for $X$ by regression.  This is
different from the corrections of section \ref{airs2cris} and
\ref{airsL1d}; there regression was used to find linear or quadratic
correction coefficients independently for each channel.

Figures \ref{dreg1} and \ref{dreg2} show residuals for direct
regression and the deconvolution translation with statistical
correction, both for apodized radiances.  As with the translation
corrections we use the 7377 profile set as the dependent and the 49
profile as the independent sets.  The residuals are roughly
comparable; the LW residual is larger at the low end of the band for
direct regression and the high end for the deconvolved translation.
Deconvolution does better in the MW, and direct regression in the
SW.  

The regression matrices show significant unwanted correlations.
Figure \ref{dreg3} shows this for the LW; the MW and SW bands are
much worse.  As noted in section \ref{airs2cris} the 7377 profile
dependent set is highly correlated.  The effective dimension is only
260, our regression is actually under-determined, and the dependent
set residuals are very small.  The independent set residuals are
larger but acceptable because the 7377 profile set approximately
spans the 49 profile test set.

One fix is to add noise.  Recall that we generate true {\airs} and
true {\cris} by convolving a common set of high-resolution radiances.
For true {\airs} we can simply add noise at the {\airs} NEdN spec.
But if we want true {\cris} radiances with noise for testing or
regression it does not work well to simply add noise at the {\cris}
NEdN spec.  This does reduce correlations but increases residuals for
the independent set significantly.  To model NEdN for the AIRS to
CrIS translation we synthesize noise at the {\airs} NEdN spec, add it
to the signal, run it through the translation, and measure it.
Similarly, for testing or regression it might be better to generate
true {\airs}, generate noise at the {\airs} NEdN spec, add this to
true {\airs}, translate the individual noise spectra to {\cris}, and
add these to true {\cris}.  The problem is getting a reference
translation for the noise, and we do not pursue that further here.

As an alternative to adding noise, we can use a form of principal
component regression.  As above, $r_a$ and $r_c$ are $m \times k$
and $n \times k$ {\airs} and {\cris} radiance sets.  Let $r_a = U_a
S_a\,V_a^T$ be a singular value decomposition with singular values
in descending order and $U_a^i$ the first $i$ columns of $U_a$.
Similarly let $r_c = U_c S_c\,V_c^T$ be a singular value
decomposition with singular values in descending order and $U_c^j$
the first $j$ columns of $U_c$.  Let $\hat r_a = (U_a^i)^T r_a$ and
$\hat r_c = (U_c^j)^T r_c$.  This gives $r_a$ and $r_c$ represented
with respect to the bases $U_a^i$ and $U_c^j$.  Since these are
orthonormal, the transpose is the inverse.  Then as before find $X$
to minimize $\|X \hat r_a - \hat r_c\|_2$ by solving $\hat r_a^T X^T
= \hat r_c^T$ for $X$ by regression.  Translating $X$ a transform on
the bases for $r_a$ and $r_c$ gives us $R = U_c^j X (U_a^i)^T$, our
{\airs} to {\cris} transform.  This is parameterized by $i$ and $j$,
the {\airs} and {\cris} basis sizes.

Figure \ref{dreg6} shows residuals and figures \ref{dreg7},
\ref{dreg8}, and \ref{dreg9} the final linear transform $R$ for the
{\cris} LW, MW, and SW bands.  We have chosen $i = j = 500$ for the
LW, $i = 500$ and $j = 320$ for the MW, and $i = j = 100$ for the SW,
to roughly balance unwanted correlation with residual size.

Note that this sort of principal component regression is not the 
same as regression after principal component (or singular vector)
filtering; for that we would take $\bar r_a = U_a^i (U_a^i)^T r_a$,
$\bar r_c = U_c^j (U_c^j)^T r_c$, find $X$ to minimize $\|X \bar r_a
- \bar r_c\|_2$, and have no need for a change of bases to apply $X$.
In practice this did not work as well as doing regression after the
change of bases.

\begin{figure} % source slackfigs a2cris_regr4.m
  \centering
  \includegraphics[height=7.5cm]{slackfigs/ap_direct_regr.png}
  \caption{mean residuals for apodized {\airs} to {\cris} direct
    regression}
  \label{dreg1}
\end{figure}

\begin{figure} % source slackfigs a2cris_test2.m
  \centering
  \includegraphics[height=7.5cm]{slackfigs/ap_decon_corr.png}
  \caption{mean residuals for apodized {\airs} to {\cris}
    reconvolution with the regression correction}
  \label{dreg2}
\end{figure}

\begin{figure} % source slackfigs a2cris_regr4.m
  \centering
  \includegraphics[height=7.5cm]{slackfigs/full_7377_LW_regr_mat.png}
  \caption{regression coefficients for the LW direct regression}
  \label{dreg3}
\end{figure}

% \begin{figure} % source slackfigs a2cris_regr4.m
%   \centering
%   \includegraphics[height=7.5cm]{slackfigs/full_7377_MW_regr_mat.png}
%   \caption{regression coefficients for the MW direct regression}
%   \label{dreg4}
% \end{figure}
% 
% \begin{figure} % source slackfigs a2cris_regr4.m
%   \centering
%   \includegraphics[height=7.5cm]{slackfigs/full_7377_SW_regr_mat.png}
%   \caption{regression coefficients for the SW direct regression}
%   \label{dreg5}
% \end{figure}

\begin{figure} % source slackfigs a2cris_regr5.m
  \centering
  \includegraphics[height=7.5cm]{slackfigs/ap_pc_direct_regr.png}
  \caption{mean residuals for apodized {\airs} to {\cris} principal
    component regression}
  \label{dreg6}
\end{figure}

\begin{figure} % source slackfigs a2cris_regr5.m
  \centering
  \includegraphics[height=7.5cm]{slackfigs/LW_pc_regr_mat.png}
  \caption{regression coefficients for the LW principal component
    regression}
  \label{dreg7}
\end{figure}

\begin{figure} % source slackfigs a2cris_regr5.m
  \centering
  \includegraphics[height=7.5cm]{slackfigs/MW_pc_regr_mat.png}
  \caption{regression coefficients for the MW principal component
    regression}
  \label{dreg8}
\end{figure}

\begin{figure} % source slackfigs a2cris_regr5.m
  \centering
  \includegraphics[height=7.5cm]{slackfigs/SW_pc_regr_mat.png}
  \caption{regression coefficients for the SW principal component
    regression}
  \label{dreg9}
\end{figure}

%---------------------------------------------------------------------
\FloatBarrier
\section{Conclusions}
\label{appcon}

[rehash intro; maybe add some notes on applications and SNOs]

\FloatBarrier
\section{Appendix}
\label{append}

We want to measure the correlation of a set of observations.  A
standard measure is the dimension of a spanning set.  An analog when
approximations are acceptable is to use the basis size needed to get
a reconstruction residual below a fixed threshold.  Let $r_0$ be an
$m \times n$ array of radiances, one row per channel and one column
per observation.  Let $r_1 = U S\,V^T$ be a singular value
decomposition with singular values in descending order, and $U_k$
the first $k$ columns of $U$.  Let $r_k = U_k U_k^T r_0$; then $r_k
\approx r_0$.  The approximation improves as $k$ increases and
becomes exact for some $k <= m$.  This is the analog of principal
component filtering using left-singular rather than eigenvectors.
This is useful as a form of compression when $k$ is small relative
to $n$.  For that case we save $U_k$ and $U_k^T r_0$ separately.
Applications include IASI radiance data and the kcarta absorption
database.

Let $B^{-1}$ be the inverse Planck function and define $d(r_1, r_2)
= \rms(B^{-1}(r_1, v) - B^{-1}(r_2, v))$, the {\rms} difference over
all channels and observations of the brightness temperatures of
radiance data.  Finally let $j$ be the smallest value such that
$d(r_0, r_j) \le T_d$, for some threshold $T_d$.  We have chosen as
$T_d = 0.02$~K.  Then $j$ is the effective dimension of our set
$r_0$.  For the 49 profile fitting set this gives $j=48$, which we
would interpret as largely uncorrelated, while for the 7377 profile
cloudy set we found $j=260$, which we would interpret as highly
correlated.

\FloatBarrier
\bibliographystyle{abbrv}
\bibliography{decon}

\end{document}

