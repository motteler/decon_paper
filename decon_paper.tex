% \documentclass[11pt]{article}
\documentclass[10pt,twocolumn]{article}
\usepackage{graphicx}
\usepackage{placeins}
\usepackage{url}

\widowpenalty10000

% LLS packages and defs

% Easy way to change geometry of the page
\usepackage{geometry}
\geometry{letterpaper,textwidth=6.9in,textheight=9in,includeheadfoot}

% Better kerning, etc.
\usepackage{microtype}

% Special font
\usepackage[T1]{fontenc}
% \usepackage[expert,altbullet]{lucidabr}
% End Special font

% Small captions
\usepackage[font=small]{caption}

% Force more floats when there is little text
\columnsep=0.2in
\setlength{\belowcaptionskip}{-1ex}
\setcounter{topnumber}{4}
\def\topfraction{.9}
\setcounter{bottomnumber}{4}
\def\bottomfraction{.9}
\setcounter{totalnumber}{4}
\def\textfraction{0}
\def\floatpagefraction{.8}
\setcounter{dbltopnumber}{4}
\def\dbltopfraction{.9}
\def\dblfloatpagefraction{.8}

% I've always hated default section sizes, always use these
\makeatletter
\renewcommand{\section}{\@startsection {section}{1}{\z@}%
                                   {-3.5ex \@plus -1ex \@minus -.2ex}%
                                   {2.3ex \@plus.2ex}%
                                   {\reset@font\large\bfseries}}
\renewcommand{\subsection}{\@startsection{subsection}{2}{\z@}%
                                     {-3.25ex\@plus -1ex \@minus -.2ex}%
                                     {1.5ex \@plus .2ex}%
                                     {\reset@font\normalsize\bfseries}}
\makeatother



\input{crisdefs.tex}

\title{AIRS deconvolution and the \\
       translation of AIRS to CrIS radiances \\ 
       with applications for the IR climate record \\
  \vspace{3mm}
  {****} DRAFT {****}\\
}

\author{Howard E.~Motteler \\
  L.~Larrabee Strow \\
  \\
  UMBC Atmospheric Spectroscopy Lab \\
  Joint Center for Earth Systems Technology \\
}

\date{\today}
\begin{document}
\maketitle

\begin{abstract}

Spectra of the earth's thermal emission as measured by the {\airs},
{\cris}, and {\iasi} hyper-spectral sounders are becoming a
significant part of the long term climate record.  These instruments
have broadly similar spatial sampling, spectral resolution, and band
spans.  However the spectral response functions differ in detail,
leading to significant differences in observed spectra.  To address
this we translate channel radiances from one sounder to another,
including simulation of the response functions of the translation
target.  We make regular use of such translations from {\airs} to
{\cris} and {\iasi} to {\cris}, and have implemented and tested
{\iasi} to {\airs} and {\cris} to {\airs} translations as well.  
Our translation from {\airs} to {\cris} has some novel features.
{\airs} is a grating spectrometer with a distinct response function
for each channel, while {\cris} is a Michaelson interferometer with
a sinc response function after calibration and corrections.  We use
our detailed knowledge of the {\airs} spectral response functions to
deconvolve {\airs} channel radiances to a resolution enhanced
intermediate representation.  This is reconvolved to {\cris} or
other instrument specifications.  The resulting translation is shown
to be more accurate than interpolation or conventional regression.

\end{abstract}

\section{Introduction}

Spectra of the earth's thermal emission as measured by the {\airs}
\cite{airs1}, {\cris} \cite{cris1,cris2}, and {\iasi} \cite{iasi1}
hyperspectral infrared sounders are becoming a significant part of
the long term climate record.  Such measurements began with {\airs}
in 2002 and should continue for the foreseeable future, given their
important role in numerical weather prediction.  These sounders are
in sun-synchronous near-polar orbits, with broadly similar spatial
sampling, spectral resolution, and spectral band spans.  However the
spectral response functions vary in detail, and this can lead to
significant differences in observed spectra.

For applications such as calibaration and validation, retrievals,
and the construction of a long term climate record we would like 
to work with single set of spectral response functions.  This is
can be done by translating channel radiances from one sounder to
another, including simulation of the response functions of the
translation target.  We make regular use of translations from
{\airs} to {\cris} and {\iasi} to {\cris}, and have implemented and
tested {\iasi} to {\airs} and {\cris} to {\airs} translations as
well.  The translations from {\iasi} includes deapodization (a form
of deconvolution) before reconvolution to the translation target,
and work very well.  Ranking these translations by accuracy in
comparison with calculated reference truth, we have {\iasi} to
{\cris}, {\iasi} to {\airs}, {\airs} to {\cris}, and finally {\cris}
to {\airs} \cite{git:decon}.  But aside from the {\airs} to {\cris}
translation the methods used are for the most part conventional.

Our translation from {\airs} to {\cris} has some novel features.
{\airs} is a grating spectrometer with a distinct response function
for each channel determined by the focal plane geometery, while
{\cris} is a Michaelson interferometer with a sinc response function
after calibration and corrections.  In section \ref{decon} we show
how to take advantage of our detailed knowledge of the {\airs}
spectral response functions (SRFs) and their overlap to deconvolve
channel radiances to a resolution-enhanced intermediate
representation, typically a $0.1$~\wn\ grid, the approximate
resolution of the tabulated {\airs} SRFs.  This intermediate
representation can then be reconvolved to an alternate instrument
specification.  Section \ref{airs2cris} gives details and validation
tests for an {\airs} to {\cris} translation, and section
\ref{airsL1d} for translation from {\airs} to an idealized grating
model.  Both translations can be further improved with a statistical
correction.  In section \ref{dregr} we consider conventional and
principal component regression for an {\airs} to {\cris}
translation, and compare this with our deconvolution-based
translation.

%---------------------------------------------------------------------
% \FloatBarrier
\section{AIRS Deconvolution}
\label{decon}

The {\airs} spectral response functions model channel response as a
function of frequency and associate channels with nominal center
frequencies.  Each {\airs} channel $i$ has an associated spectral
response function or {\srf} $\sigma_i(v)$ such that the channel
radiance $c_i = \int \sigma_i(v)r(v)\,dv$, where $r$ is radiance at
frequency $v$.  The center or peak of $\sigma_i$ is the nominal
channel frequency.

\begin{figure} % source plot_SRF2.m
  \centering
  \includegraphics[width=\linewidth]{figures/airs_sample_SRF.pdf}
  \caption{sample {\airs} spectral response functions from the low
    and high ends of the band.   The dashed line is a generalized
    Gaussian function.}
  \label{srfs1}
\end{figure}

\begin{figure} % source plot_SRF2.m
  \centering
  \includegraphics[width=\linewidth]{figures/airs_L1c_res.pdf}
  \caption{{\airs} L1c channel spacing and resolving power, $R =
    v_i/\fwhm_i$.  The relatively regular L1c channel spacing aids
    the deconvolution.}
  \label{chan1}
\end{figure}

Figure \ref{srfs1} shows typical {\airs} SRFs from the low and high
ends of the band.  Note the significant overlap in the wings.  This
can allow for a deconvolution to recover resolution beyond that of
the response functions considered individually.  The SRFs are not
necessarily symmetrical, especially at the high end of the band.
The dashed line on top of the third SRF in each group is a fit for a
generalized Gaussian, which we consider in more detail later in this
section.  Figure \ref{chan1} shows channel spacing and resolving
power for the {\airs} L1c channel set \cite{a1c:atbd}.  The variable
channel spacing and resolving power are due to the modular structure
of the focal plane.  Although not entirely regular---that is, not a
simple function of frequency---the L1c channel set is more regular
than the L1b channel set from which it is derived, and we mainly
consider the L1c set here.

Suppose we have $n$ channels and a frequency grid $\vec v$ of $k$
points spanning the union of the domains of the functions
$\sigma_i$.  The grid step size for our applications is often 0.0025
{\wn}, the default resolution for upwelling radiances calculated
using the kCompressed Atmospheric Radiative Transfer Model (\kcarta)
\cite{kcarta1}.  Let $S_k$ be an $n\times k$ array such that
$s_{i,j} = \sigma_i(v_j)/w_i$, where $w_i = \sum_j \sigma_i(v_j)$,
that is where row $i$ is $\sigma_i(v)$ tabulated at the grid $\vec
v$ and normalized so the row sum is 1.  If the channel centers are
in increasing order $S_k$ is banded, and if they are not too close
(as is the case for a few of the L1b channels) the rows are linearly
independent.  $S_k$ is a linear transform whose domain is radiance
at the grid $\vec v$ and whose range is channel radiances.  If $r$
is radiance at the grid $\vec v$, then $c = S_k r$ gives a good
approximation of the channel radiances $c_i =
\int\sigma_i(v)r(v)\,dv$.  In practice this is how we calculate
{\airs} channel radiances for the validation tests described in
subsequent sections.

% We construct $S_k$ either explicitly or implicitly from the
% {\airs} {\srf} tabulations.  The matrix $S_k$ in the former case
% is large but manageable with a banded or sparse representation.

For the {\airs} to {\cris} and other translations we are mainly
interested in the transform $S_b$ for {\srf}s at an intermediate
resolution, typically $0.1~\wn$.  This is the approximate resolution
of the {\srf} measurements and convenient for reconvolution to the
{\cris} user grid.  So let $\vec v_b = v_1,v_2,\ldots,v_m$ be a
$0.1~\wn$ grid spanning the domains of the functions $\sigma_i$.
Similar to $S_k$, let $S_b$ be an $n\times m$ array where row $i$ is
$\sigma_i(v)$ tabulated at the $\vec v_b$ grid, with rows normalized
to~1.  If $r$ is radiance at the $\vec v_b$ grid, then $c = S_b r$
is still a reasonable approximation of $\int\sigma_i(v)r(v)\,dv$.

For our application we want to start with $c$ and find $r$, that is
to deconvolve $c$ by solving $S_b r = c$ for $r$.  Since $n < m$, the
system is underdetermined.  We take the Moore-Penrose pseudoinverse
\cite{strang:linalg} of $S_b$ to get $r_0 = S_b^{-1} c$.  This gives a
minimal solution, in the sense that $||r_0||_2 \le ||r_j||_2$ for
all $r_j$ satisfying $S_b r_j = c$.  The condition number for $S_b$
as built from the L1c channels is $||S_b||_2||S_b^{-1}||_2 = 115$,
which is tolerable.

Although our main goal is to reconvolve the $0.1~\wn$ intermediate
representation to the {\cris} or other user grids, we first compare
the deconvolved radiance with reference truth from a direct
convolution of {\kcarta} radiance to the intermediate grid.  The
choice of response functions for the direct convolution is not
obvious, since the deconvolution is undoing---at least to some
extent---the effects of the {\airs} SRF convolutions.  We chose a
generalized Gaussian \cite{wiki:gauss} of the form
\[w(v, v_0, \fwhm) = 
\exp\left(-\left(\frac{(v - v_0)^2}{2c^2}\right)^{1.5}\right) \]
where $c=\fwhm / (2\sqrt{2\ln 2})$ and $v_0$ is the desired channel
center.  The exponent $1.5$ was chosen to give an approximate match
to {\airs} SRFs with the same {\FWHM} and channel centers, though
without the fine structure and variation of the measured \srf s.
Figure \ref{srfs1} shows two such generalized Gaussians paired with
the corresponding {\airs} SRFs.  We used the generalized Gaussian as
reference truth for the $0.1~\wn$ intermediate grid with ${\fwhm} =
v_i / 2000$, where $v_i$ are the grid frequencies.  This represents
a hypothetical grating spectrometer with a resolving power of 2000,
oversampled to the 0.1~\wn\ grid.  The value of 2000 was chosen to
give an approximate fit to the deconvolved radiances.  We also tried
a generalized Gaussian with a fixed {\FWHM} for values $0.4$, $0.6$,
and $0.8$ and a sinc basis with a spacing of $0.2$~\wn, all of which
gave larger residuals.

\begin{figure} % source decon_test1.m
  \centering
  \includegraphics[width=\linewidth]{figures/airs_decon_spec.pdf}
  \caption{Spectra from fitting profile 1 for direct convolution to
    the $0.1$~\wn\ grid and for deconvolved {\airs}.  We see some
    overshoot and ringing in the deconvolution.}
  \label{dspec}
\end{figure}

\begin{figure} % source decon_test1.m 
  \centering
  \includegraphics[width=\linewidth]{figures/airs_decon_zoom.pdf}
  \caption{Details from fitting profile 1 for {\kcarta}, direct
    convolution to the $0.1$~\wn\ grid, deconvolved {\airs}, and
    true {\airs}.  The deconvolution restores some detail.}
  \label{dzoom}
\end{figure}

\begin{figure} % source plot_Binv.m
  \centering
  \includegraphics[width=\linewidth]{figures/airs_decon_basis.pdf}
  \caption{sample adjacent rows for the deconvolution and L1c to L1d
    transforms}
  \label{dbasis}
\end{figure}

The {\airs} deconvolution gives a modest resolution enhancement, at
the cost of added artifacts and noise.  Figure \ref{dspec} shows the
full spectra from fitting profile~1, along with sample details from
the low and high ends of the band, for the deconvolution and direct
convolution to the intermediate grid.  In the details we see some
overshoot and ringing in the deconvolution.  Figure \ref{dzoom}
shows details of {\kcarta}, direct convolution to the
$0.1$~\wn\ grid, deconvolution, and AIRS spectra for fitting
profile~1 \cite{sarta1,sarta2}.  In the first subplot we see the
deconvolution is capturing some of the fine structure in the
{\kcarta} data that is present in the direct convolution but not in
the AIRS data.  In the second subplot we see the deconvolution (and
direct convolution) resolving a pair of close lines that are not
resolved at the {\airs} L1c resolution.  But we also see some
ringing that is not present in the direct convolution.  This is to
be expected; significant detail is lost in the convolution to
{\airs} channel radiances and this can only partially be recovered
by the deconvolution.  The artifacts are acceptable because we do
not propose using the deconvolved radiances directly; they are an
intermediate step before reconvolution to a lower resolution.

% \begin{figure} % source decon_test1.m
%   \centering
%   \includegraphics[width=\linewidth]{figures/airs_decon_diff.pdf}
%   \caption{mean and standard deviation over the 49 fitting profiles
%     for the L1c deconvolution minus direct convolution to the
%     $0.1$~\wn\ intermediate grid.  The residuals are too large to use
%     the deconvolved radiances directly.}
%   \label{ddiff}
% \end{figure}

% Figure \ref{ddiff} shows the mean and standard deviation of the
% difference of the deconvolved minus the directly convolved radiances
% for all 49 fitting profiles.  The residuals are large but mainly
% significant for understanding limitations of the deconvolution.
% The residuals can be reduced dramatically by reconvolving the
% $0.1$~\wn\ intermediate grid to a lower resolution.  We consider
% this for convolution to the {\cris} user grid in the next section.

Figure \ref{dbasis} shows a pair of typical adjacent rows of the
deconvolution matrix $S_b^{-1}$\, in the first subplot.  Row $i$ of
$S_b^{-1}$ is the weights applied to L1c channel radiances to
synthesize the deconvolved radiance $r_i$ at the intermediate grid
frequency $v_i$.  The oscillation shows we are taking the closest
AIRS channel, subtracting weighted values for channels $\pm 1$ step
away, adding weighted values for channels $\pm 2$ steps away, and so
on, with the weights decreasing quickly as we move away from $v_i$,
with eight to ten L1c channels making a significant contribution to
each deconvolution grid point.  The second subplot shows four
adjacent rows of the matrix $S_d \cdot S_b^{-1}$, which takes L1c to
L1d channel radiances.  (The L1d radiances are discussed in a later
section; here they are of interest mainly as a typical
reconvolution.)  Both matrices are banded but the bands are narrower
in the second, with three to five L1c channels contributing
significantly to each L1d channel.  The range of influence is
significant since we may want to know which L1d channels are derived
in part from the synthetic L1c channels.

%---------------------------------------------------------------------
% \FloatBarrier
\section{AIRS to CrIS translation}
\label{airs2cris}

Given {\airs} deconvolution to a $0.1~\wn$ intermediate grid,
reconvolution to the {\cris} user grid is straightforward.  For
{\cris} standard resolution the channel spacing is $0.625~\wn$ 
for the LW band, $1.25~\wn$ for the MW, and $2.5~\wn$ for the SW.
For each {\cris} band, we (1)~find the {\airs} and {\cris} band
intersection, (2) apply a bandpass filter to the deconvolved {\airs}
radiances restricting them to the intersection, with a rolloff
outside the passband, and (3) reconvolve the filtered spectra to 
the {\cris} user grid with a zero-filled double Fourier transform
\cite{git:finterp}.  The out-of-band rolloff smooths what would
otherwise be an impulse at the band edges, reducing ringing in the
translation.  

Translations are tested by comparison with calculated reference
truth.  We start with a set of atmospheric profiles and calculate
upwelling radiance at a $0.0025~\wn$ grid with {\kcarta}
\cite{kcarta1} over a band spanning the domains of the {\airs} and
{\cris} response functions.  ``True {\airs}'' is calculated by
convolving the {\kcarta} radiances with {\airs} SRFs and ``true
{\cris}'' by convolving {\kcarta} radiances to a sinc basis at the
{\cris} user grid.  True {\airs} is then translated to {\cris} to
get ``{\airs} {\cris}'', and this is compared with true {\cris}.
Figure~\ref{specLW} shows sample spectra for true {\airs},
deconvolved {\airs}, true {\cris} and {\airs} {\cris}.  Any
difference between true {\cris} and {\airs} {\cris} is hard to see
here, and in subsequent plots we mainly show explicit differences.

Comparisons are done both with and without apodization.  Hamming
apodization \cite{wiki:wind} sacrifices some resolution but gives a
significant reduction in the residuals and is convenient for many
applications.  Convolution, deconvolution, and apodization are done
with radiances, while spectra are presented and statistics are done
after conversion to brightness temperatures.

\begin{figure} % source a2cris_test1
  \centering
  \includegraphics[width=\linewidth]{figures/a2cris_spec_LW.pdf}
  \caption{true {\airs}, deconvolved {\airs}, true {\cris}, and {\airs}
    {\cris}.  Differences between true {\cris} and {\airs} {\cris} are too
    small to be visible in this figure.}
  \label{specLW}
\end{figure}

\begin{figure} % source a2cris_test1
  \centering
  \includegraphics[width=\linewidth]{figures/a2cris_diff_LW.pdf}
  \caption{Mean and standard deviation of unapodized and Hamming
    apodized {\airs} {\cris} minus true {\cris}, for the {\cris} LW
    band}
  \label{diffLW}
\end{figure}

\begin{figure} % source a2cris_test1
  \centering
  \includegraphics[width=\linewidth]{figures/a2cris_diff_MW.pdf}
  \caption{Mean and standard deviation of unapodized and Hamming
    apodized {\airs} {\cris} minus true {\cris}, for the {\cris} MW
    band.}
  \label{diffMW}
\end{figure}

\begin{figure} % source a2cris_test1
  \centering
  \includegraphics[width=\linewidth]{figures/a2cris_diff_SW.pdf}
  \caption{Mean and standard deviation of unapodized and Hamming
    apodized {\airs} {\cris} minus true {\cris}, for the {\cris} SW
    band.}
  \label{diffSW}
\end{figure}

\begin{figure} % source a2cris_test1
  \centering
  \includegraphics[width=\linewidth]{figures/a2cris_diff_all.pdf}
  \caption{Mean of apodized residuals for all three {\cris} bands,
    showing the residuals in greater detail.}
  \label{meanAll}
\end{figure}

For most tests we use a set of 49 fitting profiles spanning a wide
range of clear atmospheric conditions, initially chosen for testing
radiative transfer codes \cite{sarta1,sarta2}.  The set is largely
uncorrelated; reducing the reconstruction residual to 0.02~K
requires 48 left-singular vectors.  (Details of this correlation
measure are given in an appendix.)  For the statistical correction
described later in this section and the direct regression of section
\ref{dregr} we also use a set of 7377 radiances calculated from
all-sky (clear and cloudy) AIRS profiles spanning several
consecutive days.  This set is more correlated; reducing the
reconstruction residual to 0.02~K requires 260 left-singular
vectors.  Splitting the 7377 profile set into dependent and
independent subsets and comparing residuals from the independent
subset with residuals from the 49-profile set, residuals from the
latter are consistently larger, suggesting it makes for a stricter
test.  So for the results shown here the test or independent set is
always the 49-profile set, while for tests requiring fitting the
7377 profile is used as the dependent set.

Figures \ref{diffLW}, \ref{diffMW}, and \ref{diffSW} show the mean
and standard deviation of true {\cris} minus {\airs} {\cris} for the
49 fitting profiles, for each {\cris} band.  The Hamming apodization
gives a significant reduction in the residuals.  Figure \ref{meanAll}
summarizes results all three bands, for apodized radiances.  The
constant or DC bias is very close to zero for the apodized residuals.
The unapodized residuals are significant but both the apodized and
unapodized residuals are much less than the corresponding residuals
from conventional interpolation, as shown in the appendix.

% $0.002$~K for the LW, $-0.005$~K for the MW, and $0.001$~K for 
% the SW.

% At the low end of the {\cris} LW band there are only the two
% {\airs} ``guard channels'' below 650 {\wn}, so the rolloff is
% abrupt and we have significant ringing in the translation.

The relatively small standard deviation of the residuals suggests
some regularity, and we can see an oscillation with a period of two
channel steps in several places.  Up to this point there as been no
statistical component to our translation, beyond the choice of test
set for validation.  We feel it is important to be clear about any
steps that require statistical fitting.  That said, we can use a
simple linear correction for a significant further reduction of the
residuals.  We use the set of 7377 mostly cloudy AIRS profiles as
the dependent set and the 49 profile set as the independent or test
set.

We compare three such corrections.  These are done with a separate
regression for each {\cris} channel, and so introduce no
cross-correlations.  Let $\Ttc_i$ be true {\cris} and $\Tac_i$
{\airs} {\cris} brightness temperatures for {\cris} channel $i$,
from the dependent set.  For the bias test we subtract the mean
residual from the dependent set.  For the linear test we find $a_i$
and $b_i$ to minimize $||a_i\,\Tac_i + b_i - \Ttc_i||_2$, and for
the quadratic test weights $c_i$, $a_i$ and $b_i$ to minimize
$||c_i\,(\Tac_i)^2 + a_i\,\Tac_i + b_i - \Ttc_i||_2$.  The resulting
correction is then applied to the independent set, the 49 fitting
profiles, for comparison with true {\cris}.

Figure \ref{statLW} is a comparison of bias, linear, and quadratic
corrections for the LW band.  The linear and quadratic corrections
are nearly identical, with the quadratic coefficient very close to
zero.  Figure \ref{coefLW} shows the weights for the linear fits
from figure \ref{statLW}.  The $a$ weights are very close to 1 and
the $b$ weight to the bias.  Figures \ref{statMW} and \ref{statSW}
show the linear correction giving a similar improvement in the MW
and a small improvement in the SW, where the quadratic correction is
noticably worse.  Figure \ref{statAll1} shows the residuals for the
apodized linear correction for all three bands.  The residuals are
significantly reduced in comparison with the apodized uncorrected
radiances shown in figure \ref{meanAll} and are generally less than
NEdT (for the first fitting profile), as we show next.

\begin{figure} % source a2cris_regr2.m
  \centering
  \includegraphics[width=\linewidth]{figures/a2cris_regr_LW.pdf}
  \caption{Mean and standard deviation of LW corrected apodized
    residuals}
  \label{statLW}
\end{figure}

\begin{figure} % source a2cris_regr2.m
  \centering
  \includegraphics[width=\linewidth]{figures/a2cris_coef_LW.pdf}
  \caption{LW $a$ and $b$ weights for the linear correction $ax+b$}
  \label{coefLW}
\end{figure}

\begin{figure} % source a2cris_regr2.m
  \centering
  \includegraphics[width=\linewidth]{figures/a2cris_regr_MW.pdf}
  \caption{Mean and standard deviation of MW corrected apodized
    residuals.}
  \label{statMW}
\end{figure}

\begin{figure} % source a2cris_regr2.m
  \centering
  \includegraphics[width=\linewidth]{figures/a2cris_regr_SW.pdf}
  \caption{Mean and standard deviation of SW corrected apodized
    residuals.}
  \label{statSW}
\end{figure}

\begin{figure} % source a2cris_test2.m
  \centering
  \includegraphics[width=\linewidth]{figures/ap_decon_corr.pdf}
  \caption{Mean corrected apodized residuals for all three bands,
    showing the linear corrected apodized residuals in greater
    detail.}
  \label{statAll1}
\end{figure}

% \begin{figure} % source a2cris_test2.m
%   \centering
%   \includegraphics[width=\linewidth]{figures/a2cris_regr_all.pdf}
%   \caption{Mean corrected and uncorrected apodized residuals for all
%     three bands.}
%   \label{statAll2}
% \end{figure}

\begin{figure} % source nedn_test2.m
  \centering
  \includegraphics[width=\linewidth]{figures/a2cris_nedn.pdf}
  \caption{{\airs}, {\airs}-to-{\cris}, and {\cris} NEdN.
    Apodization reduces the {\cris} and {\airs}-to-{\cris} NEdN by a
    factor of $0.63$.}
  \label{nedn}
\end{figure}

\begin{figure} % source nedt_test1.m
  \centering
  \includegraphics[width=\linewidth]{figures/a2cris_nedt.pdf}
  \caption{{\airs}, {\airs}-to-{\cris}, and {\cris} apodized NEdT,
    and the max of {\cris} and {\airs}-to-{\cris} NEdN (shown as
    NEdT) with {\cris} NEdT shown as a reference.}
  \label{nedt}
\end{figure}

We can give a good estimate of noise equivalent differential
radiance (NEdN) for the translation by adding noise with a normal
distribution at the {\airs} NEdN to blackbody radiance at 280K and
translating this to {\cris}.  This is done repeatedly and the noise
after translation is measured.  As a check, noise before translation
is also measured and compared with the {\airs} value.  Figure
\ref{nedn} shows the measured {\airs}-to-{\cris} NEdN together with
{\airs} and {\cris} NEdN for both apodized and unapodized radiances.
The {\airs} and {\cris} values are averages over a full day, 4 Dec
2016.  NEdN for the L1c synthetic channels is interpolated.  The
first subplot of figure \ref{nedt} is NEdT for apodized radiances,
for fitting profile 1.

The {\airs} channel-to-channel NEdN variation is significant; in the
upper half of the LW and most of the MW it is of the same order as
the {\airs} and {\cris} NEdN difference.  This variation is due the
{\airs} focal plane structure and sensitivity.  The {\airs} and
{\cris} NEdN measures are both spiky when averaged over a few
minutes but the {\cris} variation is primarily uncertainty in the
noise measurement and smooths out as the time span is extended,
while the {\airs} variation is stable.  The {\airs}-to-{\cris}
translation inherits this variability; it is a significant part of
the difference between {\airs} {\cris} and true {\cris}.  For a
common record we might want to add noise on a channel-by-channel
basis to whichever NEdN value---{\airs} {\cris} or true {\cris}---is
lower.  NEdN for the combined record would then be max of the
{\airs} {\cris} and true {\cris} NEdN values, as shown in the second
subplot of figure \ref{nedt}.

In addition to the standard resolution described at the beginning 
of this section, {\cris} has a high resolution mode with a channel
spacing of $0.625~\wn$ for all three bands.  We can do a translation
from {\airs} to high-resolution {\cris} but the residuals are quite
large; {\airs} does not have sufficient resolution as a starting
point.  The high resolution mode does allow for a {\cris} to {\airs}
translation as described in \cite{git:decon}.  The residuals are
larger in the LW than for our translation from {\airs} to standard
resolution {\cris}, but may be acceptable for some applications.

%---------------------------------------------------------------------
% \FloatBarrier
\section{Translation to an idealized grating model}
\label{airsL1d}

% The nominal {\airs} resolution is 1200, though for many modules
% the real resolving power is higher.

The {\airs} deconvolution can be used for other translations.  
In this section we briefly examine reconvolution to an idealized
grating model for resolving powers of 700 and 1200.  There are
several reasons to consider such a translation.  The constant
resolving power of the L1d basis (defined below) makes it a more
natural translation target for {\airs} than the constant channel
spacing of {\cris}.  It could be considered as the next step in
regularization of the {\airs} product, following the partial
regularization from L1b to L1c.  If there were other operational
hyperspectral grating spectrometers it would be a logical target 
for a common record.

Define an {\airs} L1d basis with resolving power $R$ using the
generalized Gaussian response function of section \ref{decon} as
follows.  Let $v_0$ be the frequency of the first channel and for
$i\ge0$ $\fwhm_i = v_i / R$, $dv_i = \fwhm_i / 2$, and $v_{i+1} =
v_i + dv_i$.  As with tests of the {\airs} to {\cris} translation,
true L1c is calculated by convolving {\kcarta} radiances with
{\airs} L1c SRFs and true L1d by convolving with an L1d basis at the
desired resolving power.  L1c is translated to L1d by deconvolution
followed by reconvolution to the desired L1d basis, and this is
compared with true L1d.

Figure \ref{L1d1200} shows residuals for reconvolution to an L1d
basis with resolving power of 1200, the nominal {\airs} resolution,
and figure \ref{L1d700s} shows residuals for a resolving power of
700.  Note the different x-axes for the two figures.  The residuals
depend in part on the L1d starting channel $v_0$, and so on how the
L1c and L1d SRF peaks line up.  The residuals shown are the result
of a rough fit for $v_0$.  For a resolving power of 1200 this gave
$v_0$ equal to the first L1c channel, while for 700 it was the first
L1c channel plus $0.2$~\wn.

We see that for both the {\airs} to {\cris} and L1c to L1d
translations some resolving power is sacrificed in shifting channel
centers to a single regular function of frequency.  Residuals for a
resolving power of 1200 (figure \ref{L1d1200}) are roughly
comparable to unapodized {\cris} (figures \ref{diffLW},
\ref{diffMW}, and \ref{diffSW}) and residuals for a resolving power
of 700 (figure \ref{L1d700s}) are roughly comparable to apodized
{\cris} (figure \ref{statAll1}).  As with the {\airs} to {\cris}
translation, the L1c to L1d residuals are significantly reduced with
a linear correction.  Residuals for L1d with a resolving power of
700 after correction are comparable to residuals for apodized
{\cris} after a similar correction.

\begin{figure} % source L1d_regr1.m
  \centering
  \includegraphics[width=\linewidth]{figures/L1d_cor1_1200.pdf}
  \caption{Mean and standard deviation over the 49 fitting profiles
    for the L1c to L1d translation minus true L1d for a resolving
    power of 1200.}
  \label{L1d1200}
\end{figure}

\begin{figure} % source L1d_regr1.m
  \centering
  \includegraphics[width=\linewidth]{figures/L1d_cor1_700.pdf}
  \caption{Mean and standard deviation over the 49 fitting profiles
    for the L1c to L1d translation minus true L1d for a resolving
    power of 700.}
  \label{L1d700s}
\end{figure}

%---------------------------------------------------------------------
% \FloatBarrier
\section{Direct and principal component regression}
\label{dregr}

The {\airs} L1c to L1d translation can be done with a single 
linear transform $S_d\cdot S_c^{-1}$, where $S_c$ and $S_d$ are the
transforms taking the intermediate grid to L1c and L1d channels.
The {\airs} to {\cris} translation could also be done with a
composite transform if we use a resampling matrix rather than
double Fourier interpolation and a matrix form of the bandpass
filters.  We can get such a one-step tranform in other ways.
Suppose $r_a$ and $r_c$ are $m \times k$ and $n \times k$ {\airs}
and {\cris} radiance sets, for example true {\airs} and true {\cris}
from section \ref{airs2cris}.  We can find an $n \times m$ matrix
$X$ by regression to minimize $\|X r_a - r_c\|_2$ and use this as
our {\airs} to {\cris} transform.  We call this standard technique
``direct regression'' here.  This is different from the corrections
of section \ref{airs2cris} and \ref{airsL1d}; there regression was
used to find linear or quadratic correction coefficients
independently for each channel.

% Typically $k > m$, giving an apparently overdetermined
% system, and we solve $r_a^t X^t = r_c^t$ for $X$ by regression.

Figures \ref{dreg1} shows residuals for direct regression from
{\airs} to apodized {\cris} radiances.  As before we use the 7377
profile set as the dependent and the 49 profile as the independent
set.  The residuals are roughly comparable to the residuals from
the deconvolution translation summarized in figure \ref{statAll1}.
However the regression matrices show significant off-diagonal
correlations.  Figure \ref{dreg3} shows this for the LW; the MW and
SW bands are worse.  In addition the dependent set residuals are
very small, much less than the residuals for the independent set.
These are signs of over-fitting.  As noted in section \ref{airs2cris}
the 7377 profile dependent set is highly correlated; the effective
dimension (as defined in the appendix) is only 260.

% The LW residual is larger at the low end of the band for direct
% regression and the high end for the deconvolved translation.
% Deconvolution does better in the MW, and direct regression in 
% the SW.

\begin{figure} % source a2cris_regr4.m
  \centering
  \includegraphics[width=\linewidth]{figures/ap_dir_regr.pdf}
  \caption{Mean residuals over the 49 profile independent set for
    {\airs} to apodized {\cris} direct regression.}
  \label{dreg1}
\end{figure}

\begin{figure} % source  a2cris_regr4.m
  \centering
  \includegraphics[width=\linewidth]{figures/LW_dir_regr_mat.png}
  \caption{Regression coefficients from the 7377 profile dependent set for
  the LW direct regression.}
  \label{dreg3}
\end{figure}

One fix is to add noise.  Recall that we generate true {\airs} 
and true {\cris} by convolving a common set of high-resolution
radiances.  For true {\airs} we can simply add noise at the {\airs}
NEdN.  But for regression or testing of an {\airs} to {\cris}
translation we want {\cris} radiances with actual translated {\airs}
noise, not simply true {\cris} with noise added as per the {\cris}
NEdN specification.  The latter does reduce correlations but
increases residuals for the independent set significantly.
To model NEdN for the {\airs} to {\cris} translation of section
\ref{airs2cris} we synthesize noise at the {\airs} NEdN, add that 
to the signal, run signal plus noise through the translation, and
measure the noise.  But to get an {\airs} to {\cris} translation by
regression with added noise we need a reference translation of each
noisy {\airs} spectra.  We don't want to use the translation of
section \ref{airs2cris} for that, at least not if our goal is to
give each translation method an independent test.

\begin{figure} % source slackfigs a2cris_regr5.m
  \centering
  \includegraphics[width=\linewidth]{figures/ap_pc_regr.pdf}
  \caption{Mean residuals over the 49 profile independent set for
    {\airs} to apodized {\cris} principal component regression.}
  \label{dreg6}
\end{figure}

\begin{figure} % source slackfigs a2cris_regr5.m
  \centering
  \includegraphics[width=\linewidth]{figures/LW_pc_regr_mat.png}
  \caption{Regression coefficients from the 7377 profile dependent
    set for LW principal component regression with $i = j = 500$.}
  \label{dreg7}
\end{figure}

\begin{figure} % source slackfigs a2cris_regr5.m
  \centering
  \includegraphics[width=\linewidth]{figures/MW_pc_regr_mat.png}
  \caption{Regression coefficients from the 7377 profile dependent
    set for MW principal component regression with $i = 500$ and $j
    = 320$.}
  \label{dreg8}
\end{figure}

\begin{figure} % source slackfigs a2cris_regr5.m
  \centering
  \includegraphics[width=\linewidth]{figures/SW_pc_regr_mat.png}
  \caption{Regression coefficients from the 7377 profile dependent
    set for SW principal component regression with $i = j = 100$.}
  \label{dreg9}
\end{figure}

As an alternative to adding noise, we can use a form of principal
component regression.  As above let $r_a$ and $r_c$ be $m \times k$
and $n \times k$ {\airs} and {\cris} radiance sets.  Let $r_a = U_a
S_a\,V_a^T$ be the singular value decomposition with singular values
in descending order and $U_a^i$ the first $i$ columns of $U_a$.
Similarly let $r_c = U_c S_c\,V_c^T$ be a singular value
decomposition with singular values in descending order and $U_c^j$
the first $j$ columns of $U_c$.  Let $\hat r_a = (U_a^i)^T r_a$ and
$\hat r_c = (U_c^j)^T r_c$ be $r_a$ and $r_c$ represented with
respect to the bases $U_a^i$ and $U_c^j$.  (Since the bases are
orthonormal, the transpose is the inverse.)  Then as before find $X$
by regression to minimize $\|X \hat r_a - \hat r_c\|_2$.  This gives
us $R = U_c^j X (U_a^i)^T$, an {\airs} to {\cris} transform
parameterized by the basis sizes $i$ and $j$.

% Then as before find $X$ to minimize $\|X \hat r_a - \hat r_c\|_2$
% by solving $\hat r_a^T X^T = \hat r_c^T$ for $X$ by regression.

Note that this sort of principal component regression is not the
same as regression after principal component (or singular vector)
filtering; for that we would take $\bar r_a = U_a^i (U_a^i)^T r_a$,
$\bar r_c = U_c^j (U_c^j)^T r_c$, find $X$ to minimize $\|X \bar r_a
- \bar r_c\|_2$, and have no need for a change of bases to apply
$X$.  In practice this did not work as well as doing regression
after the change of bases.

Figure \ref{dreg6} shows residuals and figures \ref{dreg7},
\ref{dreg8}, and \ref{dreg9} the transform $R$ for the {\cris} LW,
MW, and SW bands.  We have chosen $i = j = 500$ for the LW, $i =
500$ and $j = 320$ for the MW, and $i = j = 100$ for the SW for the
basis sizes, to roughly balance unwanted correlation with residual
size.  The residuals for principal component regression are larger
than for the deconvolution translation with regression correction
and there is still significant off-diagonal correlation, especially
for the MW and SW bands.

%---------------------------------------------------------------------
% \FloatBarrier
\section{Applications}  
\label{appcon}

We have been using the {\airs} to {\cris} and {\iasi} to {\cris}
translations to analyze simultaneous nadir overpasses (SNOs)
\cite{sno1}.  We hope to create a long-term climate record spanning
observations of the individual sounders.  The initial steps are
validation with SNOs, validation of uniform geographic sampling and
subsetting, and showing consistency of global statistics.  Radiance
translation to a common subset might include added noise, as
discussed in section \ref{airs2cris}.  The radiance translations
could be explicit or implicit---that is, run once and saved as a
dataset, or run as needed on existing data.  Translations are fast
relative to read times.

The translations allows a retrieval algorithm to use the same fast
forward radiative transfer model, minimizing any differential effect
of forward model errors between the two instrument retrievals.  
An interesting potential application is to revisit the {\airs} SRF
measurements, to see if adjustments (within the original measurement
uncertainty) can reduce the translation residuals.

The translations presented here have been implemented and tested
extensively, with the associated Matlab code published on Github.
The {\airs} to {\cris} translation is I/O bound, with the in-core
time divided into roughly 25 percent for the deconvolution and 75
percent for reconvolution, taking about 22 seconds to process our
7377 profile cloudy test set on one compute node.  Calculating the
pseudo-inverse adds another 12 seconds, but that only needs to be
done when the translation parameters change.

% \FloatBarrier
\section{Appendix}
\label{append}

\subsection{Measures of correlation}

We want to measure the correlation of a set of radiances.  One such
measure is dimension of a spanning set.  For an approximation we use
the basis size needed to get reconstruction residuals below some
fixed threshold.  Let $r_0$ be an $m \times n$ array of radiances,
one row per channel and one column per observation.  Let $r_1 = U
S\,V^T$ be a singular value decomposition with singular values in
descending order and $U_k$ the first $k$ columns of $U$.  Let $r_k =
U_k U_k^T r_0$; then $r_k \approx r_0$.  The approximation improves
as $k$ increases and becomes exact for some $k <= m$.  This is the
analog of principal component filtering using left-singular rather
than eigenvectors and is useful as a form of compression when $k$ is
small relative to $n$.  For that case we save $U_k$ and $U_k^T r_0$
separately.  Applications include compression of IASI radiance data
and the {\kcarta} absorption coefficient database.

We use a threshold for equivalence that is relevant for our
applications.  Let $B^{-1}$ be the inverse Planck function and
define $d(r_1, r_2) = \rms(B^{-1}(r_1, v) - B^{-1}(r_2, v))$, the
{\rms} difference over all channels and observations of the
brightness temperatures of radiance data.  Finally let $j$ be the
smallest value such that $d(r_0, r_j) \le T_d$, for some threshold
$T_d$.  Then $j$ is the effective dimension of our set $r_0$.  Here
we have chosen $T_d = 0.02$~K.  For the 49 profile fitting set this
gives $j=48$, which we interpret as largely uncorrelated, while for
the 7377 profile cloudy set we found $j=260$, which we interpret as
highly correlated.

\subsection{Conventional interpolation}

The {\airs} to {\cris} translation via deconvolution works
significantly better than conventional interpolation.  We consider
two cases.  For the first, start with true {\airs} and interpolate
radiances directly to the {\cris} user grid with a cubic spline.
For the second, interpolate true {\airs} to the 0.1 {\wn}
intermediate grid with a cubic spline and then convolve this to the
use {\cris} user grid.  Figure~\ref{intpLW} shows interpolated
{\cris} minus true {\cris} for the LW band, without apodization.
The two-step interpolation works a little better than the simple
spline, but both residuals are significantly larger than for the
translation with deconvolution.  Results for the MW are similar,
while the unapodized comparison is less clear for the SW.  With
Hamming apodization, the residuals with deconvolution are
significantly less than interpolation for all three bands.

\begin{figure} % source a2cris_test1
  \centering
  \includegraphics[width=\linewidth]{figures/a2cris_interp_LW.pdf}
  \caption{spline interpolation, interpolation with convolution, 
    and deconvolution with convolution for the {\cris} LW band.}
  \label{intpLW}
\end{figure}

\begin{figure} % source L1d_test2.m
  \centering
  \includegraphics[width=\linewidth]{figures/CtoD_interp_diff.pdf}
  \caption{spline interpolation, interpolation with convolution, 
    and deconvolution with convolution for the {\airs} L1c to L1d
    translation with $v_0=649.822$~\wn\ and a resolving power of 700}
  \label{interpL1d}
\end{figure}

For the {\airs} L1c to L1d translation, deconvolution again works
significantly better than interpolation.  As before, we consider two
cases.  For the first, start with true {\airs} and interpolate
radiances directly to the L1d grid with a cubic spline.  For the
second, interpolate true L1c to the 0.1 {\wn} intermediate grid with
a cubic spline and convolve this to the L1d channel set.
Figure~\ref{interpL1d} shows interpolated L1d minus true L1d, for a
resolving power of 700.  The two-step interpolation works a little
better than the simple spline, but is still much larger than the
residual for translation with deconvolution.

\subsection{Source code}

% This paper, a companion talk, and 

All translation and test code discussed here is available online at
Github:

\begin{itemize}
%  \item \url{https://github.com/motteler/decon_paper}\vspace{-3mm}
   \item \url{https://github.com/strow/airs_deconv}\vspace{-3mm}
   \item \url{https://github.com/strow/iasi_decon}
\end{itemize}

% \FloatBarrier
\bibliographystyle{abbrv}
\bibliography{decon_paper}

\end{document}

