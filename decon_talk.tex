\documentclass[10pt]{beamer}
% \usetheme{Boadilla}
  \usetheme{default}

\input{crisdefs.tex}

\title{AIRS deconvolution and the \\
       translation of AIRS to CrIS radiances \\ 
}
\author{H.~E.~Motteler, L.~L.~Strow}
\institute{
  UMBC Atmospheric Spectroscopy Lab \\
  Joint Center for Earth Systems Technology \\
}
\date{\today}
\begin{document}

%----------- slide --------------------------------------------------%
\begin{frame}[plain]
\titlepage
\end{frame}
%----------- slide --------------------------------------------------%
\begin{frame}
\frametitle{AIRS spectral response functions}
\begin{itemize}

  \item Each {\airs} channel $i$ has an associated spectral response
    function or {\srf} $\sigma_i(v)$, where $v$ is frequency.

  \item Channel radiance $c_i = \int \sigma_i(v)r(v)\,dv$, where $r$
    is radiance at frequency $v$.

  \item The center or peak of $\sigma_i$ is the nominal channel
    frequency.

  \item We can approximate the {\airs} {\srf}s with a generalized
    Gaussian of the form \[w(v, v_0, \fwhm) =
    \exp\left(-\left(\frac{(v - v_0)^2}{2c^2}\right)^{1.5}\right) \]
    where $c=\fwhm / (2\sqrt{2\ln 2})$ and $v_0$ is the desired
    channel center.

  \item The exponent $1.5$ was chosen to give an approximate match
    to {\airs} {\srf}s with the same {\FWHM} and channel centers,
    though without the fine structure and variation.

\end{itemize}
\end{frame}
%----------- slide --------------------------------------------------%
\begin{frame}
\frametitle{sample SRFs and resolving power}
\begin{columns}[t]
\begin{column}{0.5\textwidth}
  \begin{centering}
  \includegraphics[width=\textwidth]{figures/airs_sample_SRF.pdf} \\
  \end{centering}\vspace{2mm}
  Sample {\airs} spectral response functions from the low and high
  ends of the band.  The dashed line is the generalized Gaussian
  function.

\end{column}
\begin{column}{0.5\textwidth}  
  \begin{centering}
  \includegraphics[width=\textwidth]{figures/airs_L1c_res.pdf}
  \end{centering}\vspace{3mm}
  {\airs} L1c channel spacing and resolving power, $R =
  v_i/\fwhm_i$.  The relatively regular L1c channel spacing aids the
  deconvolution. 

\end{column}
\end{columns}
\end{frame}
%----------- slide --------------------------------------------------%
\begin{frame}
\frametitle{the AIRS deconvolution}
\begin{itemize}

  \item Let $\vec v_b = v_1,v_2,\ldots,v_m$ be a $0.1~\wn$ grid
    spanning the domains of the functions $\sigma_i$.

  \item This is the approximate resolution of the {\srf}
    measurements and convenient for reconvolution to the {\cris}
    user grid.

  \item Let $S_b$ be an $n\times m$ array where row $i$ is
    $\sigma_i(v)$ tabulated at the $\vec v_b$ grid, with rows
    normalized to~1.

  \item Note that the $\sigma_i(v)$ are the measured {\srf}s, not
    our Gaussian approximation.

  \item If $r$ is radiance at the $\vec v_b$ grid, then $c = S_b r$
    is a reasonable approximation of $\int\sigma_i(v)r(v)\,dv$.

  \item We want to start with $c$ and find $r$, that is to
    deconvolve $c$ by solving $S_b r = c$ for $r$.  

  \item Since $m < k$ the system is underdetermined.  

\end{itemize}
\end{frame}
%----------- slide --------------------------------------------------%
\begin{frame}
\frametitle{the AIRS deconvolution}
\begin{itemize}

  \item We take the Moore-Penrose pseudoinverse of $S_b$ to get $r_0
    = S_b^{-1} c$.

  \item This gives a minimal solution, in the sense that $||r_0||_2
     \le ||r_j||_2$ for all $r_j$ satisfying $S_b r_j = c$.  

  \item The condition number for $S_b$ as built from the L1c
    channels is $||S_b||_2||S_b^{-1}||_2 = 115$, which is tolerable.

  \item Although our main goal is to reconvolve the $0.1~\wn$
    intermediate representation to the {\cris} or other user grids,
    we first compare the deconvolved radiances with reference truth
    from a direct convolution to the intermediate grid.

  \item We use the generalized Gaussian as reference truth for the
    $0.1~\wn$ intermediate grid with ${\fwhm} = v_i / 2000$, where
    $v_i$ are the grid frequencies.  

  \item This represents a hypothetical grating spectrometer with a
    resolving power of 2000, oversampled to the 0.1~\wn\ grid.

\end{itemize}
\end{frame}
%----------- slide --------------------------------------------------%
\begin{frame}
\frametitle{examples of deconvolution}
\begin{columns}[t]
\begin{column}{0.5\textwidth}
  \begin{centering}
  \includegraphics[width=\textwidth]{figures/airs_decon_spec.pdf}
  \end{centering}\vspace{3mm}
  Spectra from fitting profile 1 for direct convolution to the
  $0.1$~\wn\ grid and for deconvolved {\airs}.  We see some
  overshoot and ringing in the deconvolution.

\end{column}
\begin{column}{0.5\textwidth}  
  \begin{centering}
  \includegraphics[width=\textwidth]{figures/airs_decon_zoom.pdf}
  \end{centering}\vspace{2mm}
  Details from fitting profile 1 for kcarta, direct convolution to
  the $0.1$~\wn\ grid, deconvolved {\airs}, and true {\airs}.  The
  deconvolution restores some detail.
\end{column}
\end{columns}
\end{frame}
%----------- slide --------------------------------------------------%
\begin{frame}
\frametitle{deconvolution notes}
\begin{itemize}

  \item The {\airs} deconvolution gives a modest resolution
    enhancement, at the cost of added artifacts and noise.

  \item The deconvolution captures some fine structure that is
    present in the direct convolution but not the AIRS data and 
    can resolves lines that are merged in the {\airs} L1c spectra

  \item But we also see some ringing and overshoot that is not
    present in the direct convolution.

  \item These artifacts are acceptable because we do not propose
    using the deconvolved radiances directly; they are an
    intermediate step before reconvolution to a lower resolution.

\end{itemize}
\end{frame}
%----------- slide --------------------------------------------------%
\begin{frame}
\frametitle{AIRS to CrIS translation}
\begin{itemize}

  \item Given {\airs} deconvolution to a $0.1~\wn$ intermediate
    grid, reconvolution to the {\cris} user grid is straightforward.

  \item For the {\cris} standard resolution the channel spacing is
    $0.625~\wn$ for the LW band, $1.25~\wn$ for the MW, and
    $2.5~\wn$ for the SW.

  \item For each {\cris} band, we 
    \begin{enumerate}
       \item find the {\airs} and {\cris} band intersection, 
       \item apply a bandpass filter to the deconvolved {\airs}
         radiances restricting them to the intersection, with a
         rolloff outside the intersection, and
       \item reconvolve the filtered spectra to the {\cris} user
         grid with a zero-filled double Fourier transform
    \end{enumerate}

  \item The basic translation is from {\airs} to unapodized {\cris},
    but we will typically show both apodized and unapodizied residuals.

\end{itemize}
\end{frame}
%----------- slide --------------------------------------------------%
\begin{frame}
\frametitle{testing and validation}
\begin{itemize}

  \item Translations are tested by comparison with calculated
    reference truth.

  \item We start with a set of atmospheric profiles and calculate
    upwelling radiance at a $0.0025~\wn$ grid with kcarta over a
    band spanning the domains of the {\airs} and {\cris} response
    functions.

  \item ``True {\airs}'' is calculated by convolving the kcarta
    radiances with {\airs} SRFs and ``true {\cris}'' by convolving
    kcarta radiances to a sinc basis at the {\cris} user grid.

  \item True {\airs} is then translated to {\cris} to get ``{\airs}
    {\cris}'', and this is compared with true {\cris}.

  \item For most tests we use a set of 49 fitting profiles spanning
    a wide range of clear atmospheric conditions, initially chosen
    for testing radiative transfer codes

\end{itemize}
\end{frame}
%----------- slide --------------------------------------------------%
\begin{frame}
\frametitle{AIRS to CrIS residuals}
\begin{columns}[t]
\begin{column}{0.5\textwidth}
  \begin{centering}
  \includegraphics[width=\textwidth]{figures/a2cris_diff_LW.pdf}
  \end{centering}\vspace{3mm}
  Mean and standard deviation of unapodized and Hamming apodized
  {\airs} {\cris} minus true {\cris}, for the {\cris} LW band

\end{column}
\begin{column}{0.5\textwidth}  
  \begin{centering}
  \includegraphics[width=\textwidth]{figures/a2cris_diff_MW.pdf}
  \end{centering}\vspace{3mm}
  Mean and standard deviation of unapodized and Hamming apodized
  {\airs} {\cris} minus true {\cris}, for the {\cris} MW band

\end{column}
\end{columns}
\end{frame}
%----------- slide --------------------------------------------------%
\begin{frame}
\frametitle{AIRS to CrIS residuals}
\begin{columns}[t]
\begin{column}{0.5\textwidth}
  \begin{centering}
  \includegraphics[width=\textwidth]{figures/a2cris_diff_SW.pdf}
  \end{centering}\vspace{3mm}
  Mean and standard deviation of unapodized and Hamming apodized
  {\airs} {\cris} minus true {\cris}, for the {\cris} SW band.

\end{column}
\begin{column}{0.5\textwidth}  
  \begin{centering}
  \includegraphics[width=\textwidth]{figures/a2cris_diff_all.pdf}
  \end{centering}\vspace{5mm}
  Mean of apodized residuals for all three {\cris} bands, showing
  the apodized residuals in greater detail.

\end{column}
\end{columns}
\end{frame}
%----------- slide --------------------------------------------------%
\begin{frame}
\frametitle{AIRS to CrIS statistical correction}
\begin{columns}[t]
\begin{column}{0.5\textwidth}
  \begin{centering}
  \includegraphics[width=\textwidth]{figures/a2cris_regr_LW.pdf}
  \end{centering}\vspace{3mm}
  Mean and standard deviation of LW corrected apodized residuals.

\end{column}
\begin{column}{0.5\textwidth}
  \begin{centering}
  \includegraphics[width=\textwidth]{figures/a2cris_regr_MW.pdf}
  \end{centering}\vspace{3mm}
  Mean and standard deviation of MW corrected apodized residuals.
 
\end{column}
\end{columns}
\end{frame}
%----------- slide --------------------------------------------------%
\begin{frame}
\frametitle{AIRS to CrIS statistical correction}
\begin{columns}[t]
\begin{column}{0.5\textwidth}  
  \begin{centering}
  \includegraphics[width=\textwidth]{figures/a2cris_regr_SW.pdf}
  \end{centering}\vspace{3mm}
  Mean and standard deviation of SW corrected apodized residuals.

\end{column}
\begin{column}{0.5\textwidth}
  \begin{centering}
  \includegraphics[width=\textwidth]{figures/ap_decon_corr.pdf}
  \end{centering}\vspace{3mm}
  Mean corrected apodized residuals for all three bands, showing
  the corrected apodized residuals in greater detail.
\end{column}
\end{columns}
\end{frame}
%----------- slide --------------------------------------------------%
\begin{frame}
\frametitle{NEdN of the translation}
\begin{columns}[t]
\begin{column}{0.5\textwidth}  
  \begin{centering}
  \includegraphics[width=\textwidth]{figures/a2cris_nedn.pdf}
  \end{centering}\vspace{3mm}
  mean {\airs}, {\airs}-to-{\cris}, and mean {\cris} apodized and
  unapodized NEdN.

\end{column}
\begin{column}{0.5\textwidth}
  \begin{centering}
  \includegraphics[width=\textwidth]{figures/a2cris_nedt.pdf}
  \end{centering}\vspace{3mm}
  {\airs}, {\airs}-to-{\cris}, and {\cris} apodized NEdT,
    and the max of {\cris} and {\airs}-to-{\cris} with {\cris}
    NEdN shown as a reference.
 
\end{column}
\end{columns}
\end{frame}
%----------- slide --------------------------------------------------%
\begin{frame}
\frametitle{L1c to L1d translation}
\begin{columns}[t]
\begin{column}{0.5\textwidth}
  \begin{centering}
  \includegraphics[width=\textwidth]{figures/L1d_cor1_1200.pdf}
  \end{centering}\vspace{3mm}
  mean and standard deviation over the 49 fitting profiles for the
  L1c to L1d translation minus true L1d for a resolving power of
  1200

\end{column}
\begin{column}{0.5\textwidth}  
  \begin{centering}
  \includegraphics[width=\textwidth]{figures/L1d_cor1_700.pdf}
  \end{centering}\vspace{3mm}
  Mean and standard deviation over the 49 fitting profiles for the
  L1c to L1d translation minus true L1d for a resolving power of 700.
 
\end{column}
\end{columns}
\end{frame}
%----------- slide --------------------------------------------------%
\begin{frame}
\frametitle{principal component regression}
\begin{columns}[t]
\begin{column}{0.5\textwidth}
  \begin{centering}
  \includegraphics[width=\textwidth]{slackfigs/ap_direct_regr.png}
  \end{centering}\vspace{3mm}
  Mean residuals for apodized {\airs} to {\cris} direct regression.

\end{column}
\begin{column}{0.5\textwidth}  
  \begin{centering}
  \includegraphics[width=\textwidth]{slackfigs/full_7377_LW_regr_mat.png}
  \end{centering}\vspace{3mm}
  Regression coefficients for the LW direct regression.

\end{column}
\end{columns}
\end{frame}
%----------- slide --------------------------------------------------%
\begin{frame}
\frametitle{channel spacing and resolving power}
\begin{columns}[t]
\begin{column}{0.5\textwidth}
  \begin{centering}
  \includegraphics[width=\textwidth]{slackfigs/ap_pc_direct_regr.png}
  \end{centering}\vspace{3mm}
  Mean residuals for apodized {\airs} to {\cris} principal component
  regression.

\end{column}
\begin{column}{0.5\textwidth}  
  \begin{centering}
  \includegraphics[width=\textwidth]{slackfigs/LW_pc_regr_mat.png}
  \end{centering}\vspace{3mm}
  Regression coefficients for the LW principal component regression.
 
\end{column}
\end{columns}
\end{frame}
%----------- slide --------------------------------------------------%
\begin{frame}
\frametitle{channel spacing and resolving power}
\begin{columns}[t]
\begin{column}{0.5\textwidth}
  \begin{centering}
  \includegraphics[width=\textwidth]{slackfigs/MW_pc_regr_mat.png}
  \end{centering}\vspace{3mm}
  Regression coefficients for the MW principal component
  regression. 
 
\end{column}
\begin{column}{0.5\textwidth}  
  \begin{centering}
  \includegraphics[width=\textwidth]{slackfigs/SW_pc_regr_mat.png}
  \end{centering}\vspace{3mm}
  Regression coefficients for the SW principal component regression.

\end{column}
\end{columns}
\end{frame}
%----------- slide --------------------------------------------------%
% \begin{frame}
% \frametitle{conclusions}
% 
% \begin{itemize}
% 
%   \item with the switch to extended res, we have seen a significant
%     convergence in calibration algorithm performance 
% 
%   \item the {\noaa} ``SA-1 first'' algorithm does slightly better
%     when compared with reference truth convolved with responsivity,
%     while the {\ccast} ``ratio first'' algorithm does slightly
%     better when compared with reference truth convolved with a flat
%     passband
% 
%   \item this may be because responsivity cancels out more completely
%     in the ratio-first method
% 
%   \item because reference truth convolved with a flat passband is a
%     more conventional and non instrument-specific standard, the
%     ccast algorithm, or some similar ratio-first method, may be
%     preferable
% 
% \end{itemize}
% \end{frame}
% %----------- slide --------------------------------------------------%

\end{document}

