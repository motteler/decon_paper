
Upwelling infrared radiation as measured by the {\airs} \cite{airs1}
and {\cris} \cite{cris1,cris2} sounders is a significant part of the
long term climate record.  {\airs} and {\cris} have similar sampling
patterns \cite{git:acsamp}.  We often want to compare radiances, and
would like to treat this as a single data set for the analysis of
long term trends.  However the instruments have different spectral
resolutions, channel response functions, and band spans.  As a step
in addressing this problem we consider the translation of channel
radiances from {\airs} to standard resolution {\cris}.

In addition to our {\airs} to {\cris} translation we make regular
use of an {\iasi} to {\cris} translation for evaluating simultaneous
nadir overpasses (SNOs) \cite{sno1}, and have implemented and tested
{\iasi} to {\airs} and {\cris} to {\airs} translations as well.  The
translations from {\iasi} includes deapodization (a form of
deconvolution) before reconvolution to the translation target, and
work very well.  Ranking these translations by accuracy in
comparison with calculated reference truth, we have {\iasi} to
{\cris}, {\iasi} to {\airs}, {\airs} to {\cris}, and finally {\cris}
to {\airs} \cite{git:decon}.  But aside from {\airs} to {\cris} the
methods used are for the most part conventional.

Our translation from {\airs} to {\cris} has some novel features.
{\airs} is a grating spectrometer with a distinct response function
for each channel determined by the focal plane geometery, while
{\cris} is a Michaelson interferometer with a sinc response
function after calibration and corrections.  In section \ref{decon}
we show how to take advantage of our detailed knowledge of the
{\airs} spectral response functions (SRFs) and their overlap to
deconvolve channel radiances to a resolution-enhanced intermediate
representation, typically a $0.1$~\wn\ grid, the approximate
resolution of the tabulated {\airs} SRFs.  
This intermediate representation can then be reconvolved to an
alternate instrument specification.  Section \ref{airs2cris} gives
details and validation tests for the {\airs} to {\cris} translation
and section \ref{airsL1d} for translation to an idealized grating
model.  For both cases deconvolution followed by reconvolution is
shown to work significantly better than conventional interpolation.
Both methods can be further improved with a statistical correction.
In section \ref{dregr} we consider a purely statistical approach to
such translations, and compare this with deconvolution.
