\documentclass[11pt]{article}
\usepackage{graphicx}
\usepackage{placeins}
\usepackage{url}
\usepackage{cite}
\usepackage{geometry}
\geometry{top=2cm}


% acronyms for text or math mode
\newcommand {\ccast} {\mbox{\small CCAST}}
\newcommand {\cris} {\mbox{\small CrIS}}

\newcommand {\airs} {\mbox{\small AIRS}}
\newcommand {\iasi} {\mbox{\small IASI}}
\newcommand {\idps} {\mbox{\small IDPS}}
\newcommand {\nasa} {\mbox{\small NASA}}
\newcommand {\noaa} {\mbox{\small NOAA}}
\newcommand {\nstar} {\mbox{\small STAR}}
\newcommand {\umbc} {\mbox{\small UMBC}}
\newcommand {\uw}   {\mbox{\small UW}}

\newcommand {\fft}  {\mbox{\small FFT}}
\newcommand {\ifft} {\mbox{\small IFFT}}
\newcommand {\fir}  {\mbox{\small FIR}}
\newcommand {\fov}  {\mbox{\small FOV}}
\newcommand {\for}  {\mbox{\small FOR}}
\newcommand {\ict}  {\mbox{\small ICT}}
\newcommand {\ils}  {\mbox{\small ILS}}
\newcommand {\igm}  {\mbox{\small IGM}}
\newcommand {\opd}  {\mbox{\small OPD}}
\newcommand {\rms}  {\mbox{\small RMS}}
\newcommand {\zpd}  {\mbox{\small ZPD}}
\newcommand {\ppm}  {\mbox{\small PPM}}
\newcommand {\srf}  {\mbox{\small SRF}}
\newcommand {\sdr}  {\mbox{\small SDR}}

\newcommand {\ES} {\mbox{\small ES}}
\newcommand {\SP} {\mbox{\small SP}}
\newcommand {\IT} {\mbox{\small IT}}
\newcommand {\SA} {\mbox{\small SA}}

\newcommand {\ET} {\mbox{\small ET}}
\newcommand {\FT} {\mbox{\small FT}}

\newcommand {\wn} {\mbox{cm$^{-1}$}}

% abbreviations, mainly for math mode
\newcommand {\real} {\mbox{real}}
\newcommand {\imag} {\mbox{imag}}
\newcommand {\atan} {\mbox{atan}}
\newcommand {\obs}  {\mbox{obs}}
\newcommand {\calc} {\mbox{calc}}
\newcommand {\sinc} {\mbox{sinc}}
\newcommand {\psinc} {\mbox{psinc}}
\newcommand {\std} {\mbox{std}}

% symbols, for math mode only
\newcommand {\lmax} {L_{\mbox{\tiny max}}}
\newcommand {\vmax} {V_{\mbox{\tiny max}}}

\newcommand {\tauobs} {\tau_{\mbox{\tiny obs}}}
\newcommand {\taucal} {\tau_{\mbox{\tiny calc}}}
\newcommand {\Vdc}  {V_{\mbox{\tiny DC}}}

\newcommand {\rIT} {r_{\mbox{\tiny\textsc{ict}}}}
\newcommand {\rES} {r_{\mbox{\tiny\textsc{es}}}}
\newcommand {\robs} {r_{\mbox{\tiny obs}}}

\newcommand {\rITobs} {r_{\mbox{\tiny\textsc{ict}}}^{\mbox{\tiny obs}}}
\newcommand {\rITcal} {r_{\mbox{\tiny\textsc{ict}}}^{\mbox{\tiny cal}}}

\newcommand {\ITmean} {\langle\mbox{\small IT}\rangle}
\newcommand {\SPmean} {\langle\mbox{\small SP}\rangle}


\newcommand {\reply} {\mbox{\small REPLY}}

\begin{document}

\title{Author's Response to Reviewer \#1 }

\author{AIRS Deconvolution and the \\
       Translation of AIRS to CrIS Radiances \\ 
       with Applications for the IR Climate Record \\
       \\
       Howard~E.~Motteler, L.~Larrabee~Strow}

\maketitle

\section{General Remarks}

We thank the reviewers for their thoughtful comments, and have 
tried to incorporate all suggestions and respond to all questions.
Section~2 below includes reviewer's comments followed by our
responses.

The discussion of reference truth for the deconvolution (starting on
page 3 line 30 RHS of the original submission) was not clear.  We
have updated the labels in figures 3 and 4 from ``gauss'' to ``decon
ref'' to better indicate we are showing the deconvolution reference
truth.  We updated the associated discussion to emphasize that the
deconvolution reference truth is simply a check of the deconvolution
step---we don't need it to do the deconvolution or for subsequent
reconvolution to CrIS or other targets.

We corrected the relationship of standard deviation and full-width
half-max (FWHM), $s=\fwhm / (2\sqrt{2\ln 2})$ in the original
submission, to $s=\fwhm / (2\sqrt{2}\,(\ln 2)^{1/(2p)})$.  The
latter is correct for the generalized Gaussian.  The difference is
small for the range of values $p$ we used.  As noted above this does
not effect the accuracy of the translations, just (to a very small
degree) the consistency of reference truth for the deconvolution and
the basis functions for our L1d ``idealized grating model''.

\section{Response to Comments}

\begin{itemize}

\item p.1 Line.14 r.h.s.: It would be useful to introduce here,
  rather than on p.3 and p.6, that the CrIS instrument has two modes
  of operation, the nominal and full spectral resolution (NSR and
  FSR, respectively).  It should also be mentioned that CrIS was in
  the NSR from launch in 10/28/2011 through Dec. 2014 or Dec. 2015
  (after bit-trim mask upgrade, depending how much detail you want
  to cover here.

  \reply: We agree this info should be added, but in section 3 the
  (AIRS to CrIS translation) rather than the introduction.  There is
  a limit to what fits in a sensible introduction---for example we
  do not get into other significant details, such as AIRS L1b vs
  L1c, there.  The key idea of the paper is that we can get a modest
  resolution enhancement from the AIRS deconvolution and we do say
  that in the introduction.  As it turns out this is good enough for
  NSR but probably not FSR, but again the CrIS section is the
  natural place for that discussion.
  
% For this paper the CrIS and L1d translations are sample
% applications, not the main story.

\item p.1 Line.60 r.h.s.: ``The SRFs are not necessarily
  symmetrical, especially at the high end of the band.''  The SRFs
  are not necessarily symmetrical due to fringing in the AIRS
  entrance filters, especially at the high end of the band” is
  repetitive and can be combined into one sentence.

  \reply: Right, we fixed this.

\item p.1 Section 1: It might be best to simply include a table of
  significant instrument attributes for AIRS, IASI, and CrIS. Things
  like type (grating or interferometer), spectral sampling,
  resolution, launch dates, number of channels, and NEDN at selected
  frequencies.

  \reply: This would be interesting, but we can't really justify
  discussing IASI here, and AIRS and CrIS are different enough that
  the differences are not easily summed up in a table.  A short
  descriptive paragraph, expanding a bit on what we have now, is
  probably best.  The suggested table would fit better in in a paper
  dealing with the many questions beyond response functions that
  come up in building a common data set.

\item p.2 Fig.1: It might be worth plotting this on a vertical log
  scale. The real AIRS SRF’s have a long tail that is not adequately
  demonstrated in a linear vertical scale. The “significant overlap”
  mentioned on p.1 was by design as a Nyquist sampled spectrum was
  desired (as is IASI and CrIS) but there is additional significant
  overlap caused by the long-tail. These details should be briefly
  mentioned so the reader is not misled into thinking that AIRS can
  be represented by a simple Gaussian function in the equation that
  follows later on this page.

  \reply: It's interesting to look at the SRFs this way, and as you
  note the Gaussian approximation does not have the long tails of
  the tabulated SRFs.  We've added a comment to this effect.  But we
  aren't proposing to use the approximation as an alternative to the
  tabulated SRFS.  The applications are (1) reference truth for the
  deconvolved radiances at a different spacing and resolving power,
  used for understanding the deconvolution, and (2) basis functions
  for the idealized grating model of section~4.

\item p.2 line 25 r.h.s.: on the first uses of the $||r||_2$ it
  would be helpful to let the reader know this is a Euclidian norm
  or even more obvious, a root sum square.

  \reply: We've added a note to that effect.  The notation used in
  the paper is taken from the Wikipedia articles for the
  Moore-Penrose inverse and mathematical norm, and we added both as
  citations.

\item p.2 line 37-39 r.h.s.: The $c$ in this equation is not the
  same as the $c$ used above (in $S_b r = c$). Suggest that a
  different symbol be used to avoid confusing the reader.

  \reply: Agreed, we replaced this with $s$.

\item p.3 line 45-50 r.h.s.: The CrIS NSR mode will ultimately
  become a historical oddity for S-NPP. It is irrelevant for JPSS-1
  to JPSS-4 (ultimately to be known as NOAA-21, 22, 23, 24 if they
  survive launch) as the NSR will no longer be processed. Thus, the
  authors need to justify why they would degrade the entire AIRS
  2002 to 2016 and S-NPP/NOAA-20+ record from 2016 to 2030’s instead
  of potentially finding an alternative higher resolution grid. For
  example, consider the loss of spectral information in the AIRS
  carbon monoxide spectral domain when going from AIRS SRFs to CrIS
  NSR – the CO band information is completely lost. Similar loss
  occurs in the 2390 cm-1 R-branch that affects lower tropospheric
  temperature information.

  \reply: We agree, medium to long term.  But in the meantime we
  have the significant overlap of AIRS and CrIS NSR data to work
  with.  The effective resolution of deconvolved AIRS does not take
  us quite to CrIS FSR for the MW and SW bands.  You can still do a
  translation but the residuals are relatively large in comparison
  with those shown in the paper.

  One solution might be to pick an intermediate resolution for CrIS,
  for example 0.6 \wn\ in the MW, that roughly corresponds to the
  AIRS effective resolution, and we've added a note to that effect.
  This is easy to do for both regular CrIS processing and our AIRS
  to CrIS translation.  In both cases we have an intermediate
  representation---sensor grid for CrIS and our deconvolution grid
  for AIRS---that can be resampled to any nominal resolution we
  like.  We've expanded the discussion of NSR and FSR at the end of
  the CrIS translation section and used this topic as a lead-in to
  the next section, translation to an idealized grating model.

\item p.3 Fig.4: I had a tough time understanding exactly what was
  plotted here and I think the text and caption could be improved.
  I believe AIRS is the original AIRS radiance. Kcarta is the 0.0025
  cm-1 LBL representation, the Gauss and decon are a comparison of
  the 0.1 cm-1 intermediate resolution where Gauss is derived from
  direct convolution of k-Carta and decon is derived by the
  pseudo-inverse process – but I am not sure if I have Gauss/Decon
  flipped. It would be best if the plot labels and text were
  identified explicitly.

  \reply: Agreed.  We've update the figure labels and associated
  discussion.  In figures 3 and 4 the old label ``gauss'' is now
  ``decon ref'' (reference truth for the deconvolution) and the old
  ``decon'' is now ``AIRS decon'' (deconvolved AIRS radiances).

\item p.4 line 34 l.h.s.: given that Hamming apodization is
  reversible there cannot be any loss in spectral resolution. Thus,
  the appearance of the radiances in unapodized (or deapodized since
  there is self- apodization effects with CrIS) versus Hamming
  apodized only has to do with non-local ILS effects (i.e.,
  side-lobes of the SRF) and not resolution. The 15 micron band line
  spacing is a resonance that is aliased with the CrIS OPD, such
  that the sinc() function produces a distortion of that band (i.e.,
  the peaks and troughs are exaggerated).

  \reply: We agree there is no loss in interferometric resolution
  with an invertible apodization.  Hamming apodization does reduce
  resolving power $R = v_i/ \fwhm_i$, since it increases $\fwhm$
  while decreasing the side-lobes.  We changed ``resolution'' to
  ``resolving power'' and set Barnet et al. 2000 as noted below as
  our first citation for apodization, for the sentence in question.

% For the deconvolution-based translation apodization is a separate,
% distinct step after deconvolution and reconvolution or resampling
% to the final user grid, 

% and our ``true Cris'' minus ``AIRS CrIS'' residuals are
% significantly smaller with apodization.  This is true for almost
% every sort of test we do, for example comparing calibrated
% measured and calculated radiances.
  
\item p.4 line.34 l.h.s.: The AIRS SRF is a modified Gaussian that
  is mostly a local function. It seems odd that the authors would
  propose convert AIRS to a unapodized SRF since the sinc() function
  is a lon-local function. That is, all AIRS channels would be
  necessary to produce a CrIS sinc() function. Alternatively, if one
  were to convert the AIRS channels to CrIS Hamming channels then
  each CrIS channel would be reconstructed from a small and local
  set of AIRS channels. The Hamming apodized radiances could then be
  converted to unapodized via a transformation as suggested in
  Barnet, C.D., J.M. Blaisdell and J.  Susskind 2000. Practical
  methods for rapid and accurate computation of interferometric
  spectra for remote sensing applications. IEEE
  Trans. Geosci. Remote Sens. v.38 p.169-183.

  \reply: The short answer is that the deconvolution-based
  translation is a two-step process, and the deconvolution is
  independent of the final translation target.  The Gaussian basis
  is used to convolve kcarta radiances in an attempt to get a rough
  reference truth for the deconvolution; it is not actually used 
  for either the deconvolution or subsequent reconvolution to CrIS
  radiances.

  For what we called ``direct regression'' discussed in section 5,
  we did try a one-step translation from AIRS to apodized CrIS,
  taking advantage of the relatively limited span of the apodized
  response functions.  If that had worked better we could invert the
  Hamming apodization and get unapodized radiances that way.  We use
  that technique in other contexts, for example to get unapodized
  CrIS radiances from an apodized fast model.

\item p.4 line 39-56 l.h.s.: this discussion adds confusion. It
  looks like the 49 set will be used for training and the 7377 is
  the independent set. Seems like this could be said most
  succinctly.

  \reply: Agreed, and we've rewritten this paragraph.  The 49
  profile set is always the test or independent set, as it makes for
  a more strict test.

\item p.4 line 54-59 r.h.s. Could the authors give some explanation
  as to the physical basis for the statistical correction they are
  about to discuss. I was completely lost as to why this should be
  necessary. One concern came to mind and I will share it here. It
  seems like this ``ringing'' or ``regularity'' could be caused by
  trying to compute a sinc() function from a regularly spaced
  intermediate Gaussian (that also has side-lobes). The fact that it
  is diminished by Hamming says it might be an artifact caused by
  the sinc() side-lobes or band edges. Maybe this is a naïve idea
  (or at least it might give you a clue how to improve this
  discussion), but why did you not consider 0.1 cm-1 boxcars for the
  intermediate spectrum that would have perfect localization?

  \reply: the motivation for the statistical correction was simply
  that the residuals showed some regular structure and the standard
  deviation was relativly small.  The deconvolution is an imperfect
  process, so it seemed plausible that a correction might help.

  We emphasize again that the Gaussian basis was used to convolve
  kcarta radiances in an attempt to get at least a rough reference
  truth for the deconvolution; it is not actually used for either
  the deconvolution or subsequent reconvolution to CrIS radiances.
  We tried a number of functions for this reference truth, including
  the 0.1 cm-1 boxcar you suggest.  0.2 cm-1 gave slightly smaller
  residuals than 0.1 cm-1, but these were still larger than with the
  generalized Gaussian.  We don't go into this in the paper because
  we believe there is nothing surprising about getting significant
  imperfections from the deconvolution.  The interesting thing is
  that most of this dissapears with reconvolution, leaving us with a
  translation that works well in comparison with other approaches.
  
\item p.6 line 48-55 l.h.s. and p.8 line 52-55: For most
  applications the noise spectral correlation is something that
  needs to be specified.  Since you have used linear transformations
  you have potentially altered the amplitude of the random component
  of NEDN and potentially added significant correlation (see Barnet
  2000 reference for a discussion of this in the application of
  apodization). Any data assimilation or retrieval application would
  be strongly impacted by this correlation. Given that one
  application of this methodology is for climate data records, it is
  worth discussing if the NEDN can be accurately translated.  The
  AIRS, IASI, and CrIS have similar information content in the
  signal-to-noise; however, if only the signal (radiance spectra)
  are transformed and not the noise than this process could, in
  fact, alter the information content such that AIRS, IASI, and CrIS
  cannot be combined in a meaningful manner. This thought is tied to
  the comments given on p.3 line 45-50 r.h.s. above.

  \reply: This is a good point, but a problem for any sounder 
  to sounder translation.  I think the best we can do here is to
  mention this and cite Barnet et al. 2000 again for this matter.
  We could model the correlation and add correlated noise to the
  CrIS radiances to better match the AIRS to CrIS translation, if
  that seems desirable.  But that's a matter for future work.

\item p.6 line 47-48 r.h.s.: see comment for p.3 line 45-50 r.h.s. –
  this seems worthy of a bit more discussion given that this
  approach has been advertised as a solution for a climate data
  record.

  \reply: Agreed, we've expanded this, as discussed earlier.

\item p.10 line 38-58 l.h.s. (and p.4 line 47-49 r.h.s): this is a
  trivial point. The conventional interpolation would be the wrong
  thing to do. For interferometers an exact transformation could be
  done via cosine transforms (presumably what you are doing for the
  IASI-to-CrIS transformations). The issue here is what is the best
  manner to transform a grating instrument with non-ideal SRF’s to
  an interferometer system. Maybe this discussion is leading towards
  an alternate solution – converting all instruments to an idealized
  SRF which preserves the signal-to-noise as best as possible. This
  would be an ideal localized function that is Nyquist sampled and
  has minimum noise correlation (similar to the discussion on p.7
  l.h.s., but with the noise component discussed also). The AIRS is
  nearly a localized, Nyquist sampled, spectrum whereas the CrIS and
  IASI unapodized (or more properly deapodized) spectrum is
  not-ideal and introduces significant spectral distortion. The CrIS
  and IASI apodized spectra is nearly localized by the Hamming
  function has significant (1\%) side-lobes that diminish somewhat
  slowly (see Barnet 2000 for discussion).

  \reply: Some related applications do start from conventional
  interpolation of AIRS data.  For example the JPL algorithm for
  small frequency shifts uses a cubic spline augmented with a
  statistical fit for the derivatives near the initial channel
  frequencies.  And in the past we have used spline interpolation to
  a regular intermediate grid before reconvolution via the cosine
  transform; that is, we used spline interpolation in exactly the
  same way we use deconvolution here.
  
  The ``L1d basis'' was our take on the idealized SRF you suggest.
  It started out as an attempt at finding an approximate, relatively
  simple model for the AIRS SRFs.  We agree this is an interesting
  alternative as the target for a common data set.  A key point of
  the paper that the AIRS deconvolution could be useful for more
  than one final target.

% It might be hard to convince the interferometric community that
%  something like that is desireable as a common target, though.

\end{itemize}
\end{document}

