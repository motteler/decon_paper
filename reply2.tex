\documentclass[11pt]{article}
\usepackage{graphicx}
\usepackage{placeins}
\usepackage{url}
\usepackage{cite}
\usepackage{geometry}
\geometry{top=2cm}


% acronyms for text or math mode
\newcommand {\ccast} {\mbox{\small CCAST}}
\newcommand {\cris} {\mbox{\small CrIS}}

\newcommand {\airs} {\mbox{\small AIRS}}
\newcommand {\iasi} {\mbox{\small IASI}}
\newcommand {\idps} {\mbox{\small IDPS}}
\newcommand {\nasa} {\mbox{\small NASA}}
\newcommand {\noaa} {\mbox{\small NOAA}}
\newcommand {\nstar} {\mbox{\small STAR}}
\newcommand {\umbc} {\mbox{\small UMBC}}
\newcommand {\uw}   {\mbox{\small UW}}

\newcommand {\fft}  {\mbox{\small FFT}}
\newcommand {\ifft} {\mbox{\small IFFT}}
\newcommand {\fir}  {\mbox{\small FIR}}
\newcommand {\fov}  {\mbox{\small FOV}}
\newcommand {\for}  {\mbox{\small FOR}}
\newcommand {\ict}  {\mbox{\small ICT}}
\newcommand {\ils}  {\mbox{\small ILS}}
\newcommand {\igm}  {\mbox{\small IGM}}
\newcommand {\opd}  {\mbox{\small OPD}}
\newcommand {\rms}  {\mbox{\small RMS}}
\newcommand {\zpd}  {\mbox{\small ZPD}}
\newcommand {\ppm}  {\mbox{\small PPM}}
\newcommand {\srf}  {\mbox{\small SRF}}
\newcommand {\sdr}  {\mbox{\small SDR}}

\newcommand {\ES} {\mbox{\small ES}}
\newcommand {\SP} {\mbox{\small SP}}
\newcommand {\IT} {\mbox{\small IT}}
\newcommand {\SA} {\mbox{\small SA}}

\newcommand {\ET} {\mbox{\small ET}}
\newcommand {\FT} {\mbox{\small FT}}

\newcommand {\wn} {\mbox{cm$^{-1}$}}

% abbreviations, mainly for math mode
\newcommand {\real} {\mbox{real}}
\newcommand {\imag} {\mbox{imag}}
\newcommand {\atan} {\mbox{atan}}
\newcommand {\obs}  {\mbox{obs}}
\newcommand {\calc} {\mbox{calc}}
\newcommand {\sinc} {\mbox{sinc}}
\newcommand {\psinc} {\mbox{psinc}}
\newcommand {\std} {\mbox{std}}

% symbols, for math mode only
\newcommand {\lmax} {L_{\mbox{\tiny max}}}
\newcommand {\vmax} {V_{\mbox{\tiny max}}}

\newcommand {\tauobs} {\tau_{\mbox{\tiny obs}}}
\newcommand {\taucal} {\tau_{\mbox{\tiny calc}}}
\newcommand {\Vdc}  {V_{\mbox{\tiny DC}}}

\newcommand {\rIT} {r_{\mbox{\tiny\textsc{ict}}}}
\newcommand {\rES} {r_{\mbox{\tiny\textsc{es}}}}
\newcommand {\robs} {r_{\mbox{\tiny obs}}}

\newcommand {\rITobs} {r_{\mbox{\tiny\textsc{ict}}}^{\mbox{\tiny obs}}}
\newcommand {\rITcal} {r_{\mbox{\tiny\textsc{ict}}}^{\mbox{\tiny cal}}}

\newcommand {\ITmean} {\langle\mbox{\small IT}\rangle}
\newcommand {\SPmean} {\langle\mbox{\small SP}\rangle}


\newcommand {\reply} {\mbox{\small REPLY}}

\begin{document}

\title{Author's Response to Reviewer \#2 }

\author{AIRS Deconvolution and the \\
       Translation of AIRS to CrIS Radiances \\ 
       with Applications for the IR Climate Record \\
       \\
       Howard~E.~Motteler, L.~Larrabee~Strow}

\maketitle

\section{General Remarks}

We thank the reviewers for their thoughtful comments and have tried
to incorporate or respond to all suggestions and questions.  In the
remainder of this section we summarize the main changes we've made.
Section~2 consists of reviewer's comments followed by our responses.

The discussion of reference truth for the deconvolution (starting on
page 3 line 30 RHS of the original submission) was not clear.  We
have updated the labels in figures 3 and 4 from ``gauss'' to ``decon
ref'' to better indicate we are showing the deconvolution reference
truth.  We updated the associated discussion to emphasize that the
deconvolution reference truth is intended as validation of the
deconvolution---we don't need it to do the deconvolution or for
subsequent reconvolution to CrIS or other targets.

We corrected the relationship of standard deviation and full-width
half-max (FWHM), $s=\fwhm / (2\sqrt{2\ln 2})$ in the original
submission, to $s=\fwhm / (2\sqrt{2}\,(\ln 2)^{1/(2p)})$.  The
latter is correct for the generalized Gaussian.  The difference is
small for the range of values $p$ we used.  As noted above this does
not effect the accuracy of the translations, just (to a very small
degree) reference truth for the deconvolution and the basis
functions for our L1d ``idealized grating model''.

The section on regression translation has been revised and
shortened.  We dropped the paragraph on adding noise since we don't
show those results.  The direct and principal component regression
matrices are now shown only for the MW, enough for an overview.  
The regression matrices are are now for apodized radiance, matching
the residuals.  The interpolation section in the appendix has been
revised to show apodized residuals and to emphasize the advantage of
using both the source and target response functions.

\section{Response to Comments}

\begin{itemize}
  \item 1.~Page 2 right column, line 26, $||r_0||_2$ should be
    $||r_0||^2$, there are several other places need to be
    corrected.

    \reply: $||r_0||^2$ would typically be the square of the
    Euclidian distance.  That would be harmless in our inequality
    $||r_0||_2 \le ||r_j||_2$ but not what we want for condition
    number $||S_b||_2||S_b^{-1}||_2$ in the next sentence.
    $||r||_2$ is a common notation for the $L^2$ norm, and for
    vectors is simply Euclidian distance.  The notation used in the
    paper is taken from the Wikipedia articles for the Moore-Penrose
    inverse and mathematical norm, and we've added both as
    citations.

%   Strang (our other reference for generalized inverse and a common
%   linear algebra text and reference) uses $||r||$ for Euclidian
%   distance and $||r||^2$ for distance squared.  Since we don't
%   want to imply the square of the norm, we have left this notation
%   as-is.

  \item 2.~Page 2, right column, lines 36-39 in the generalized
    Gaussian equation, please replace $c$ with other symbol.

    \reply: Agreed, we replaced this with $s$.

  \item In Fig 3 subplots 2 and 3: why do we see some overshoot
    and ringing in the deconvolution, especially at the shortwave
    CO$_2$ absorption lines?

    \reply: The ringing or overshoot at 2310~{\wn} is caused by a
    change in the L1c channel spacing from 1.02~{\wn} to 0.92~{\wn}
    at that point.  We added a comment in the paper to that effect.
    These jumps occur at the AIRS focal plane module boundaries.
    The first subplot of figure 2 shows channel spacing as a
    function of frequency, and the jump at 2310~{\wn} is one of
    several such discontinuities.  

    Of course then you can ask why should a change in the channel
    spacing cause ringing or overshoot.  We see this at band edges
    and bigger channel gaps, too---any time the L1c channels are not
    regularly spaced.  This is simply a limitation of the
    deconvolution.  We start with channel radiances and our
    tabulated SRFs $S_b$ and deconvolve $c$ by solving $S_b r = c$
    for $r$.  Since $n < m$, the system is underdetermined, and we
    might get some $r$ that we don't want.  We tried some different
    smoothing constraints and found the Moore-Penrose property of
    giving $r_0$ such that $||r_0||_2 \le ||r_j||_2$ for all $r_j$
    satisfying $S_b r_j = c$ worked best.  There are still some
    problems but these are reduced significantly by the subsequent
    reconvolution, and so probably acceptable for our application.

  \item 4. The AIRS L1c channel spacing and resolving power R is
    around 1200 in Fig 2 (after deconvolution), why the direct
    convolution choose a resolving power of 2000 instead of 1200?
    The generalize Gaussian function could be adjusted to use a
    resolving power of 1200 to better match the deconvolved
    radiances.

    \reply: The direct convolution is reference truth for the
    deconvolution, and is now labeled ``decon ref'' in the plots.
    We choose 2000 because this is an approximate match to the
    resolving power of the deconvolution.  Resolving power of the
    deconvolution is a property of the deconvolution---we just want
    to measure it.  We compared the deconvolution with reference
    truth for a range of resolving powers from 1200 to 2400 and
    chose 2000 because the residuals do not decrease much beyond
    that point.

    We don't use the Gaussian basis for the AIRS deconvolution or
    the AIRS to CrIS translation---it is used as a measure to see
    how well the deconvolution step alone works.  The deconvolution
    reference truth for a resolving power of 1200 is a very close
    match to the AIRS radiances before deconvolution.

  \item 5. Page 4 right column, line 46, ``The constant or DC bias
    is...''  What is DC?

    \reply: DC is direct current.  We dropped this because we are
    doing statistics, not signal processing, and now just say ``the
    constant bias''.

  \item 6. Page 4 right column, lines 53-54, ``Up to this point
    there as been no statistical...'' please correct this sentence

    \reply: OK, fixed.

  \item 7. Page 5, regarding the NEdN. I am not clear how do you
    measure the AIRS-to-CrIS NEdN. If you know AIRS NEdN at 280 K,
    how do you translate that NEdN to CrIS observation? I don’t
    understand the sentence: ``This is done repeatedly and the noise
    after translation is measured''.

    \reply: We modified the sentence to say ``We can give a good
    estimate of noise equivalent differential radiance (NEdN) for
    the translation by adding noise with a normal distribution at
    the {\airs} NEdN to blackbody radiance at 280K, translating this
    to {\cris}, and measuring the noise of the translation.''

    For a little more detail, the steps are (1) generate blackbody
    radiance at 280 K, at the AIRS channel set, (2) generate a set
    of $n$ normally distributed noise vectors at the AIRS NEdN spec
    and add these to the noise-free 280 K spectra, giving a set of
    $n$ noisy spectra, (3) translate this set from AIRS to CrIS, and
    (4) measure the standard deviation of the translation.  As a
    sanity check we also measure the standard deviation before the
    translation, to verify that it agrees with our AIRS NEdN spec.

\end{itemize}

\end{document}

